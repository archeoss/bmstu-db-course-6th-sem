\section{Исследовательский раздел}

В данном разделе приведены описание эксперимента и технические характеристики устройства, на котором проводилось измерение времени работы программного обеспечения, а также результаты замеров времени.

\subsection{Постановка эксперимента}

В данном подразделе представлены цель, описание и результаты эксперимента.

\subsubsection{Цель эксперимента}

Целью эксперимента является сравнение времени, требуемого для получения сильносвязанных данных о тесте Тьюринга в двух базах данных~\texttt{SurrealDB} и \texttt{PostgresQL}.

\subsubsection{Описание эксперимента}
Cравнить занимаемое время для получения данных для различных баз данных можно при помощи бенчмарков --- специальных функций, котороые проводят серии различных испытыний с записью производительности системы для дальнейшего их сравнения.
Для \texttt{SurrealDB} в рамках бенчмаркинга были написаны запросы, утилизирующие её графовую составляющую.
В то же время запросы к~\texttt{PostgresQL} применяют множественные JOIN-запросы ввиду сильносвязанности данных.

Для замера производительности двух ралзичных баз данных при выполнении запросов будет использоваться библиотека \texttt{Criterion}, функции которой использовались для определения эффективности запросов по времени.

\subsubsection{Технические характеристики}

Ниже приведены технические характеристики устройства, на котором было проведено тестирование ПО:

\begin{itemize}
    \item[$-$] Операционная система: Arch Linux~\cite{arch-linux} 64-bit;
	\item[$-$] Количество ядре: 4 физических и 8 логических ядер;
    \item[$-$] Оперативная память: 16 Гб, DDR4;
    \item[$-$] Процессор: 11th Gen Intel\textsuperscript{\tiny\textregistered} Core\textsuperscript{\tiny\texttrademark} i5-11320H @ 3.20 ГГц~\cite{i5}.
\end{itemize}

Во время тестирования устройство было нагружено только встроенными приложениями окружения, а также непосредственно системой тестирования.

\subsubsection{Результаты эксперимента}

В таблицах \ref{tbl:experiment1} - \ref{tbl:experiment3} представлены результаты поставленного эксперимента, где сравнивается время исполнения в зависимости от количесвта сущностей в базе данных.

\begin{table}[H]
	\centering
	\caption{Результаты сравнения времени, для запросов к \texttt{PostgresQL} и \texttt{SurrealDB} (количество элементов - 100 единиц)}
	\label{tbl:experiment1}
	\resizebox{\textwidth}{!}{%
		\begin{tabular}{|l|l|l|}
			\hline
			\textbf{\begin{tabular}[c]{@{}c@{}}Количество \\ запросов\end{tabular}} & \textbf{\texttt{SurrealDB}, мс} & \textbf{\texttt{PostgresQL}, мс}  \\ \hline
			1 & 58732 & 45812 \\ \hline
			5 & 255079 & 224391 \\ \hline
			10 & 527629 & 473282 \\ \hline
			25 & 1428489 & 1262168 \\ \hline
			100 & 5021948 & 4624102 \\ \hline
		\end{tabular}%
	}
\end{table}

\begin{table}[H]
	\centering
	\caption{Результаты сравнения времени, для запросов к \texttt{PostgresQL} и \texttt{SurrealDB} (количество элементов - 1000 единиц)}
	\label{tbl:experiment2}
	\resizebox{\textwidth}{!}{%
		\begin{tabular}{|l|l|l|}
			\hline
			\textbf{\begin{tabular}[c]{@{}c@{}}Количество \\ запросов\end{tabular}} & \textbf{\texttt{SurrealDB}, мс} & \textbf{\texttt{PostgresQL}, мс}  \\ \hline
			1 & 63418 & 68719 \\ \hline
			5 & 284249 & 301548 \\ \hline
			10 & 577324 & 598925  \\ \hline
			25 & 1531132 & 1762836 \\ \hline
			100 & 6145253 & 7843628 \\ \hline
		\end{tabular}%
	}
\end{table}

\begin{table}[H]
	\centering
	\caption{Результаты сравнения времени, для запросов к \texttt{PostgresQL} и \texttt{SurrealDB} (количество элементов - 5000 единиц)}
	\label{tbl:experiment3}
	\resizebox{\textwidth}{!}{%
		\begin{tabular}{|l|l|l|}
			\hline
			\textbf{\begin{tabular}[c]{@{}c@{}}Количество \\ запросов\end{tabular}} & \textbf{\texttt{SurrealDB}, мс} & \textbf{\texttt{PostgresQL}, мс} \\ \hline
			1 & 76418 & 95213 \\ \hline
			5 & 328253 & 414436 \\ \hline
			10 & 663635 & 737435  \\ \hline
			25 & 1931132 & 2435168 \\ \hline
			100 & 7296236 & 9396236 \\ \hline
		\end{tabular}%
	}
\end{table}

В рисунках \ref{img:plot1} - \ref{img:plot3} представлены визулизация результатов поставленного эксперимента в виде графиков.

\img{100mm}
{plot1}
{Зависимость времени от количества запросов (количество элементов --- 100)}

\img{100mm}
{plot2}
{Зависимость времени от количества запросов (количество элементов --- 1000)}

\img{100mm}
{plot3}
{Зависимость времени от количества запросов (количество элементов --- 5000)}

\clearpage

\subsection*{Вывод}
Результаты эксперимента показали, что в то время как при небольшом количестве сущностей в базе данных \texttt{PostgresQL} показывает лучшие результаты, то при увеличении количества сущностей в базе данных \texttt{SurrealDB} показывает лучшие результаты.
Из данного наблюдения можно сделать следующие выводы:
\begin{itemize}
    \item[$-$] Реляционные базы данных показывают лучшие результаты при небольшом количестве сущностей в базе данных или в случае, если данные не сильно связаны;
	\item[$-$] Графовые базы данных показывают лучшие результаты при большом количестве сущностей в базе данных в случаях, если данные сильно связаны.
\end{itemize}

Даже несмотря на то, что в данном эксперименте были использованы только две базы данных, можно сделать вывод, что графовые базы данных показывают лучшие результаты при большом количестве сущностей в базе данных в случаях, если данные сильно связаны.
Учитывая подобный выигрыш по времени графовых баз данных над реляционными, можно сделать вывод, что графовые базы данных являются более оптимальным решением для таких задач как:
Социальные сети, рекомендательные системы, анализ связей в бизнесе и прочих задач, где основой данных являются связи между сущностями.

