\section{Исследовательский раздел}

В данном разделе приведены описание исследования и технические характеристики устройства, на котором проводилось измерение времени работы программного обеспечения, а также результаты замеров времени.

\subsection{Постановка задачи исследования}

\subsubsection{Цель исследования}

Целью исследования является сравнение времени, требуемого для получения сильносвязанных данных о тесте Тьюринга в двух базах данных \texttt{SurrealDB} и \texttt{PostgresQL}.

\subsubsection{Описание исследования}
Cравнить занимаемое время для получения данных для различных баз данных можно при помощи бенчмарков --- специальных функций, котороые проводят серии различных испытыний с записью производительности системы для дальнейшего их сравнения.
Для \texttt{SurrealDB} в рамках бенчмаркинга были написаны запросы, утилизирующие её графовую составляющую.
В то же время запросы к \texttt{PostgreSQL} применяют множественные JOIN-запросы ввиду сильносвязанности данных.

На листинге~\ref{lst:postgres} приведён SQL-запрос для \texttt{PostgreSQL}, по которому проводилось сравнение. 
Следует отметить, что в случае PostgresQL связь между таблицами осуществляется при помощи внешних ключей (\texttt{Foreign Key}). 
Во всем остальном, таблицы аналогичны тем, что написаны для \texttt{SurrealDB}.

\clearpage

\begin{lstlisting}[
		caption={SQL запрос для \texttt{PostgreSQL}.},
		label={lst:postgres}
	]
    SELECT 
    		Session.SessionID,
        Interrogator.Name, 
        Computer.Model, 
        Human.Name, 
        Question.QuestionText,
        Answer.AnswerText,
        Verdict.Correct
    FROM Session
        JOIN Interrogator ON Session.SessionID = Interrogator.SessionID
        JOIN Computer ON Session.SessionID = Computer.SessionID
        JOIN Human ON Session.SessionID = Human.SessionID
        JOIN Question ON Session.SessionID = Question.SessionID
        JOIN Answer ON Session.SessionID = Answer.SessionID
        JOIN Verdict ON Session.SessionID = Verdict.SessionID
\end{lstlisting}

Аналогичный запрос написанный для \texttt{SurrealDB}, утилизирующий графовую составляющую базы данных:

\begin{lstlisting}
SELECT id,
			<-participateIn<-human.name,
			<-participateIn<-interroagtor.name, 
			<-participateIn<-computer.model, 
			->includes->answer.text, 
			->includes->question.text, 
			->includes->verdict.correct 
						FROM session;
\end{lstlisting}

Для замера производительности двух ралзичных баз данных при выполнении запросов использовались библиотека \texttt{Criterion}, функции которой использовались для определения эффективности запросов по времени.
Количество итераций и алгоритм анализа выборок контролируются библиотекой~\cite{criterion-anal} и работают следующим образом:
\begin{itemize}
	\item[$-$] Перед началом самих замеров проводится этап <<разогрева>>. 
	На этом этапе процедура выполняется повторно, чтобы дать операционной системе и процессору время адаптироваться к новой рабочей нагрузке.
	\item[$-$] Измерения проводятся на нескольких выборках. Каждая выборка состоит из одной или нескольких итераций процедуры.

	По мере выполнения измерений количество итераций выборки увеличивается. Количество итераций вычисляется следующим образом:

	\texttt{Количество итераций = [d, 2d, 3d, ... Nd]}

	Где N - общее количество выборок, а d - коэффициент, рассчитанный на основе приблизительной оценки времени итерации, измеренной в течение периода разогрева, который используется для масштабирования количества итераций в соответствии с настроенным временем измерения --- 60 секунд.
	
	\item[$-$] Первым шагом в анализе является классификация выбросов. 
		Каждая выборка классифицируется с использованием модифицированной версии метода Тьюки~\cite{tukey}.
		Во-первых, межквартильный диапазон (\texttt{IQR}) рассчитывается исходя из разницы между 25 - м и 75 - м процентилями. 
		В методе Тьюки значения меньше (25 - й процентиль - 1,5 * IQR) или больше (75 - й процентиля + 1,5 * IQR) считаются выбросами. 

		На втором шаге среди выборок вычисляется ряд метрик. 
		В данном исследовании используется среднее значение полученных выборок.
\end{itemize}

\subsubsection{Технические характеристики}

Ниже приведены технические характеристики устройства, на котором будет проведено исследование:

\begin{itemize}
    \item[$-$] Операционная система: Arch Linux~\cite{arch-linux} 64-bit;
		\item[$-$] Количество ядре: 4 физических и 8 логических ядер;
    \item[$-$] Оперативная память: 16 Гб, DDR4;
    \item[$-$] Процессор: 11th Gen Intel\textsuperscript{\tiny\textregistered} Core\textsuperscript{\tiny\texttrademark} i5-11320H @ 3.20 ГГц~\cite{i5}.
\end{itemize}

Во время тестирования устройство было подключено к сети электропитания и было нагружено только встроенными приложениями окружения, а также непосредственно системой тестирования.

\subsubsection{Результаты исследования}

В таблицах \ref{tbl:experiment1} - \ref{tbl:experiment3} представлены результаты поставленного эксперимента, где сравнивается время исполнения в зависимости от количесвта сущностей в базе данных.

\clearpage
\begin{table}[ht!]
	\centering
	\caption{Результаты сравнения времени, для запросов к \texttt{PostgreSQL} и \texttt{SurrealDB} (количество элементов - 100 единиц)}
	\label{tbl:experiment1}
	\resizebox{\textwidth}{!}{%
		\begin{tabular}{|l|l|l|}
			\hline
			\textbf{\begin{tabular}[c]{@{}c@{}}Количество \\ запросов\end{tabular}} & \textbf{\texttt{SurrealDB}, мс} & \textbf{\texttt{PostgresQL}, мс}  \\ \hline
			1 & 58732 & 45812 \\ \hline
			5 & 255079 & 224391 \\ \hline
			10 & 527629 & 473282 \\ \hline
			25 & 1428489 & 1262168 \\ \hline
			100 & 5021948 & 4624102 \\ \hline
		\end{tabular}%
	}
\end{table}

\begin{table}[ht!]
	\centering
	\caption{Результаты сравнения времени, для запросов к \texttt{PostgreSQL} и \texttt{SurrealDB} (количество элементов - 1000 единиц)}
	\label{tbl:experiment2}
	\resizebox{\textwidth}{!}{%
		\begin{tabular}{|l|l|l|}
			\hline
			\textbf{\begin{tabular}[c]{@{}c@{}}Количество \\ запросов\end{tabular}} & \textbf{\texttt{SurrealDB}, мс} & \textbf{\texttt{PostgresQL}, мс}  \\ \hline
			1 & 63418 & 68719 \\ \hline
			5 & 284249 & 301548 \\ \hline
			10 & 577324 & 598925  \\ \hline
			25 & 1531132 & 1762836 \\ \hline
			100 & 6145253 & 7843628 \\ \hline
		\end{tabular}%
	}
\end{table}

\begin{table}[ht!]
	\centering
	\caption{Результаты сравнения времени, для запросов к \texttt{PostgreSQL} и \texttt{SurrealDB} (количество элементов - 5000 единиц)}
	\label{tbl:experiment3}
	\resizebox{\textwidth}{!}{%
		\begin{tabular}{|l|l|l|}
			\hline
			\textbf{\begin{tabular}[c]{@{}c@{}}Количество \\ запросов\end{tabular}} & \textbf{\texttt{SurrealDB}, мс} & \textbf{\texttt{PostgresQL}, мс} \\ \hline
			1 & 76418 & 95213 \\ \hline
			5 & 328253 & 414436 \\ \hline
			10 & 663635 & 737435  \\ \hline
			25 & 1931132 & 2435168 \\ \hline
			100 & 7296236 & 9396236 \\ \hline
		\end{tabular}%
	}
\end{table}

\clearpage

На рисунках \ref{img:plot1} - \ref{img:plot3} представлены визулизация результатов поставленного эксперимента в виде графиков.

\img{100mm}
{plot1}
{Зависимость времени от количества запросов (количество элементов --- 100)}

\img{100mm}
{plot2}
{Зависимость времени от количества запросов (количество элементов --- 1000)}

\img{100mm}
{plot3}
{Зависимость времени от количества запросов (количество элементов --- 5000)}

\clearpage

\subsection*{Вывод}
Результаты эксперимента показали, что, в то время как при небольшом количестве сущностей в базе данных \texttt{PostgresQL} показывает лучшие результаты, при увеличении количества сущностей в базе данных \texttt{SurrealDB} показывает лучшие результаты.
Из данного наблюдения можно сделать следующий вывод:
графовые базы данных показывают лучшие в сравнении с реляционными базами данных результаты при большом количестве сущностей в базе данных.

