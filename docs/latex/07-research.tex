\section{Исследовательский раздел}

В данном разделе приведены технические характеристики устройства, на котором проводилось измерение времени работы программного обеспечения, а также результаты замеров времени.

\subsection{Постановка эксперимента}

Целью эксперимента является провести анализ скорости работы алгоритма генерации изображения с использованием алгоритма Z-буфером.

\subsubsection{Технические характеристики}


Ниже приведены технические характеристики устройства, на котором было проведено тестирование ПО:

\begin{itemize}
    \item[$-$] Операционная система: Arch Linux \cite{arch} 64-bit.
	  \item[$-$] Количество ядре: 4 физических и 8 логических ядер.
    \item[$-$] Оперативная память: 16 Гб.
    \item[$-$] Процессор: 11th Gen Intel\textsuperscript{\tiny\textregistered} Core\textsuperscript{\tiny\texttrademark} i5-11320H @ 3.20 ГГц\cite{i5}.
\end{itemize}

 Во время тестирования устройство было нагружено только встроенными приложениями окружения, а также непосредственно системой тестирования.

\subsubsection{Результаты эксперимента}

Для исследования зависимости времени обработки изображения от числа объектов на сцене, использовались объекты с фиксированным количеством граней, каждый объект имел освещенную и затененную части. Количество объектов менялось на сцене от 100 до 1000 с шагом 100. Результаты проведенного исследования представлены.

\img{100mm}{exp1}{График зависимости времени отрисовки от числа объектов}

Как видно из графика, время визуализации сцены зависит от количества объектов линейно, независимо от числа граней.

Следующим этапом исследования разработанной программы является исследование зависимости времени построения сцены от количества граней при фиксированном количестве объектов. В ходе эксперимента количество граней менялось по закону $X * 2^n$, где $X$ --- количество граней базовой модели, для наглядности были использованы различные модели. 
\img{100mm}{exp2}{График зависимости времени отрисовки от числа граней}

Из проведенного эксперимента можно сделать вывод, что время визуализации сцены линейно зависит от числа граней объектов, поскольку сам график соответствует графику увеличения граней.

\subsection*{Вывод}
В данном разделе приведены результаты работы программного
обеспечения и проведен эксперимент с использованием библиотеки Criterion,
функции которой использовались для определения эффективности работы
программы по времени.

Результаты эксперимента совпали с ожидаемыми, так как в ходе
эксперимента было установлено, что время работы увеличивается линейно с увеличением
количества объектов на сцене и количеством ребер у объекта.
