\specsection{ОПРЕДЕЛЕНИЯ}
% \addcontentsline{toc}{specsection}{ОПРЕДЕЛЕНИЯ}

В настоящей расчетно-пояснительной записке применяют следующие термины с соответствующими определениями.\\

\begin{description}
	\item{Natural Language Processing} --- <<Обработка текстов на естественном языке>> относится к области компьютерных наук, а точнее к области искусственного интеллекта или ИИ, связанной с предоставлением компьютерам возможность понимать текст и произносимые слова почти так же, как люди.
   НЛП сочетает в себе вычислительную лингвистику — моделирование человеческого языка на основе правил — со статистическими моделями, машинным обучением и моделями глубокого обучения. Вместе эти технологии позволяют компьютерам обрабатывать человеческий язык в виде текстовых или голосовых данных и «понимать» его полное значение, включая намерения и чувства говорящего или пишущего.~\cite{nlp}
	\item{ACID} --- в контексте обработки транзакций аббревиатура ACID относится к четырем ключевым свойствам транзакции: атомарность (Atomicity), непротиворечивость (Consistency), изоляция (Isolation) и устойчивость (Durability).~\cite{gdb-def}.
	\item{NoSQL} --- подход к проектированию баз данных, который фокусируется на предоставлении механизма хранения и извлечения данных, который моделируется средствами, отличными от табличных отношений, используемых в реляционных базах данных. Вместо типичной табличной структуры реляционной базы данных базы данных, NoSQL содержит данные в рамках одной структуры данных. Поскольку такая конструкция нереляционной базы данных не требует схемы, она обеспечивает быструю масштабируемость для управления большими и обычно неструктурированными наборами данных~\cite{nosql}.
	\item{NewSQL} --- класс современных реляционных СУБД, которые стремятся обеспечить ту же масштабируемую производительность, что и \texttt{NoSQL}, для рабочих нагрузок чтения-записи \texttt{OLTP}, сохраняя при этом гарантии \texttt{ACID} для транзакций.
		Другими словами, эти системы хотят достичь той же масштабируемости, что и СУБД \texttt{NoSQL} 2000х годов, но при этом сохранить реляционную модель (с \texttt{SQL}) и поддержку транзакций устаревших СУБД 1970x -- 1980х годов~\cite{newsql}.
	\item{Universally Unique Identifier} --- <<Универсальный уникальный идентификатор>> --- метка, используемая для уникальной идентификации ресурса среди всех других ресурсов этого типа. Является 128-битным значением, обычно представленным в виде 36-буквенной строки~\cite{uuid}.
\end{description}
