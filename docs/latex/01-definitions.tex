\specsection{ОПРЕДЕЛЕНИЯ}
% \addcontentsline{toc}{specsection}{ОПРЕДЕЛЕНИЯ}

В настоящей расчетно-пояснительной записке применяют следующие термины с соответствующими определениями.\\

\begin{description}
	\item{Natural Language Processing} --- <<Обработка текстов на естественном языке>> относится к области компьютерных наук, а точнее к области искусственного интеллекта или ИИ, связанной с предоставлением компьютерам возможность понимать текст и произносимые слова почти так же, как люди.
   НЛП сочетает в себе вычислительную лингвистику — моделирование человеческого языка на основе правил — со статистическими моделями, машинным обучением и моделями глубокого обучения. Вместе эти технологии позволяют компьютерам обрабатывать человеческий язык в виде текстовых или голосовых данных и «понимать» его полное значение, включая намерения и чувства говорящего или пишущего.~\cite{nlp}
	\item{ACID} --- В контексте обработки транзакций аббревиатура ACID относится к четырем ключевым свойствам транзакции: атомарность (Atomicity), непротиворечивость (Consistency), изоляция (Isolation) и устойчивость (Durability).~\cite{gdb-def}.
	\item{Graph Database} --- база данных, использующая структуры графов для семантических запросов с узлами, ребрами и свойствами для представления и хранения данных.~\cite{gdb-def}. 
\end{description}
