\include{preamble.inc}

\begin{document}

\pagenumbering{gobble}

    \normalsize

    \pagenumbering{arabic}
\includepdf[pages={1}]{RPZ_title.pdf}

% \specsection{РЕФЕРАТ}
% \setcounter{page}{2}
%
% Расчетно-пояснительная записка 56 с., 15 рис., 23 ист., 1 прил.
%
% КОМПЬЮТЕРНАЯ ГРАФИКА, АЛГОРИТМЫ УДАЛЕНИЯ НЕВИДИМЫХ ЛИНИЙ, Z-БУФЕР, ЗАКРАСКА, ОСВЕЩЕНИЕ, СЦЕНА, ТЕЛА ВРАЩЕНИЯ.
%
% Целью работы является разработка программы, позволяющее создавать композиции тел вращения.
%
% Для визуализации сцены использовался алгоритм с Z-буфером, а для представления модели тела вращения использовалась поверхностная модель, представленная в виде списка ребер.
%
% В процессе работы были проанализированы различные алгоритмы, методы представления, закраски геометрических моделей на сцене. 
%
% Проведено исследование быстродействия программы при различном количестве создания объектов на сцене с фиксированным количеством ребер. 
% Также с фиксированным количеством объектов, но с различным количеством ребер. 
% Из результатов исследования следует, что время отрисовки сцены увеличивается как при увеличении количества объектов на сцене, так и при увеличении количества ребер.
\newpage
\renewcommand{\contentsname}{\normalsize\bfseries\centering СОДЕРЖАНИЕ}
    \tableofcontents
    \normalsize

\specsection{ОПРЕДЕЛЕНИЯ}
% \addcontentsline{toc}{specsection}{ОПРЕДЕЛЕНИЯ}

В настоящей расчетно-пояснительной записке применяют следующие термины с соответствующими определениями.\\

\begin{description}
	\item{Natural Language Processing} --- <<Обработка текстов на естественном языке>> относится к области компьютерных наук, а точнее к области искусственного интеллекта или ИИ, связанной с предоставлением компьютерам возможность понимать текст и произносимые слова почти так же, как люди.
   НЛП сочетает в себе вычислительную лингвистику — моделирование человеческого языка на основе правил — со статистическими моделями, машинным обучением и моделями глубокого обучения. Вместе эти технологии позволяют компьютерам обрабатывать человеческий язык в виде текстовых или голосовых данных и «понимать» его полное значение, включая намерения и чувства говорящего или пишущего.~\cite{nlp}
	\item{ACID} --- В контексте обработки транзакций аббревиатура ACID относится к четырем ключевым свойствам транзакции: атомарность (Atomicity), непротиворечивость (Consistency), изоляция (Isolation) и устойчивость (Durability).~\cite{gdb-def}.
	\item{Graph Database} --- база данных, использующая структуры графов для семантических запросов с узлами, ребрами и свойствами для представления и хранения данных.~\cite{gdb-def}. 
\end{description}

\specsection{ОБОЗНАЧЕНИЯ И СОКРАЩЕНИЯ}
% \addcontentsline{toc}{specsection}{ОБОЗНАЧЕНИЯ И СОКРАЩЕНИЯ}

В настоящей расчетно-пояснительной записке применяют следующие сокращения и обозначения.\\

\begin{description}	
	\item{NLP} --- <<Natural Language Processing>>.
	\item{ACID} --- <<Atomicity, Consistency, Isolation, Durability>>.
	\item{SQL} --- <<Structured Query Language>>.
	\item{GDB} --- <<Graph Database>>
	\item{ИИ} --- <<Искусственный Интеллект>>
\end{description}


\specsection{ВВЕДЕНИЕ}
% \addcontentsline{toc}{specsection}{Введение}

Тест Тьюринга --- это способ определения способности машины производить интеллектуальные действия, неотличимые от действий человека. В современном мире тесты Тьюринга стали одним из ключевых инструментов для определения искусственного интеллекта. Эти тесты позволяют определить, насколько хорошо компьютер может имитировать разговор человека.
Для проведения теста Тьюринга используется программное обеспечение, которое имитирует человеческое поведение и должно убедить эксперта в том, что он общается с живым человеком. Результаты проведения тестов могут быть использованы для разработки и улучшения алгоритмов искусственного интеллекта. Это позволяет разработчикам алгоритмов искусственного интеллекта улучшать свои продукты и технологии.

Примерами таких алгоритмов являются алгоритмы обработки естественных языков (Natural Language Processing --- NLP). В общих словах --- это совокупность методов и техник, которые позволяют компьютерам анализировать, понимать и генерировать естественный язык. NLP используется в ряде приложений, включая автоматический перевод, распознавание речи и анализ текста. Анализ результатов тестирования поможет в будущем улучшить данные модели, позволяя избегать различные грамматические, орфографические и смысловые ошибки. 

Модели от команды OpenAI и многие другие играют важную роль в развитии искусственного интеллекта. GPT-3 неплохо справляется с созданием художественной литературы, поэзии, пресс-релизов, кода, музыки, шуток, технических руководств и новостных статей. Возможно, как предполагает Чалмерс (2020, Other Internet Resources), GPT-3 «предлагает потенциальный бездумный путь к общему искусственному интеллекту». Но, конечно, GPT-3 даже не близок к прохождению теста Тьюринга: на глобальном уровне — учитывая значения нескольких предложений, абзацев или двустороннего диалога — становится очевидным, что GPT-3 не понимает, о чем говорит. У него нет здравого смысла (Common Sense) или способности отслеживать объекты во время обсуждения. Разработанная база данных для хранения и обработки результатов тестов Тьюринга --- это хороший инструмент для создания более развитых систем искусственного интеллекта. 

Цель данной работы --- разработать базу данных для хранения результатов проведения тестов Тьюринга. База данных должна содержать информацию о тестируемых программах, экспертах, результатах проведения тестов и другие данные, необходимые для анализа результатов.

Чтобы достигнуть поставленной цели, требуется решить следующие задачи: 
\begin{itemize}
    \item[$-$] Определить структуру базы данных и ее таблиц.
    \item[$-$] Разработать модели данных для каждой таблицы.
    \item[$-$] Написать запросы для вставки, обновления и удаления данных в таблицах.
    \item[$-$] Реализовать функционал для получения результатов проведенных тестов, с возможностью фильтрации и сортировки по различным полям.
    \item[$-$] Разработать интерфейс пользователя для удобного использования базы данных. 
\end{itemize}

\section{Аналитический раздел}

В данном разделе описана структура теста Тьюринга.  
Представлен анализ способов хранения данных и систем управления базами данных, оптимальных для решения поставленной задачи. Описаны  
проблемы кэшированных данных и представлены методы их решения.

\subsection{Формализация задачи}
Тест Тьюринга --- это метод оценки способности машины производить интеллектуальные действия, сравнивая ее поведение с поведением человека в решении задач. Тест заключается в том, что человек задает вопросы человеку и компьютеру, а затем пытается определить, от кого пришли ответы. Если компьютер может убедительно имитировать поведение человека, то он считается способным производить интеллектуальные действия.

\img{100mm}
{turing.png}
{Тест Тьюринга}

Тьюринг в своей работе~\cite{10.1093/mind/LIX.236.433} описывает следующий вид игры. Предположим, что у нас есть человек, машина и эксперт. Эксперт находится в комнате, отделенной от другого человека и машины. Цель игры состоит в том, чтобы эксперт определил, кто из двух является человеком, а кто машиной. Эксперт знает человека и машину по меткам «X» и «Y» --- но, по крайней мере в начале игры, не знает, кто из них человек и кто -- машина --- и в конце игры он должен сказать либо «X --- это человек, а Y --- машина», либо «X --- это машина, а Y --- человек».
Эксперту разрешается задавать человеку и машине вопросы следующего вида: «Скажите, пожалуйста, X, играет ли X в шахматы?» Кто бы из машины и другого человека ни был X, он должен отвечать на вопросы, адресованные X. Цель машины состоит в том, чтобы попытаться заставить эксперта ошибочно заключить, что машина --- это другой человек; цель другого человека состоит в том, чтобы попытаться помочь эксперту правильно идентифицировать машину.~\cite{sep-turing-test}

Следует отметить, что во времена Тьюринга, было ограничение, что ответы поступали через ограниченные врменные рамки, поскольку время ответа компьютера было гораздо больше, чем у человека. 
Сегодня это ограничение сохраняется, однако из-за обратного: реакция компьютера быстрее, чем реакция человека.

\subsection{Способы хранения данных}

Для решения задачи теста Тьюринга необходимо хранить данные о человеке, машине и эксперте. Также необходимо хранить данные о заданных вопросах и полученных ответах. Хранить данные, на какие вопросы были даны какие ответы. Поскольку в конце теста выносится вердикт о том, является ли отвечающий машиной или человеком, необходимо также хранить какие ответы были даны в каком порядке и каким актором. Эти данные должны быть доступны для обработки и сравнения в процессе игры.

Один из способов хранения данных --- это использование реляционных баз данных. В этом случае можно создать таблицы для каждого объекта (люди, машины и эксперты, вопросы и ответы, и т.д.) и связать их отношениями. Например, таблицы «Person», «Machine» и «Interrogator» могут быть связаны через внешние ключи. Это является надежным и проверенным способом хранения данных.

Однако более эффективный способ хранения данных --- это использование графовых баз данных (GDB), таких как SurrealDB или Neo4j. В этом случае каждый объект может быть представлен узлом графа, а отношения между объектами --- ребрами графа.~\ref{img:graph}

\img{100mm}
{graph}
{Представление базы данных в виде графа}

Ключевым понятием системы является граф (или ребро, или взаимосвязь). Граф связывает элементы данных в хранилище с набором узлов и ребер, причем ребра представляют отношения между узлами. Отношения позволяют напрямую связывать данные в хранилище и во многих случаях извлекать их с помощью одной операции. Базы данных графов удерживают отношения между данными в качестве приоритета. Запрашивать отношения быстро, потому что они постоянно хранятся в базе данных. Отношения можно интуитивно визуализировать с помощью баз данных графов, что делает их полезными для сильно взаимосвязанных данных.~\cite{graph-exp}.
Поскольку графовая модель данных более естественным образом отображает связи между объектами, это делает ее более подходящей для задач, связанных с анализом связей и отношений между данными. В графовых базах данных нет необходимости использовать сложные JOIN-запросы, что может существенно упростить запросы к данным.

Графовые базы данных также обеспечивают быстрый доступ к данным по отношениям, что делает их эффективными при работе с глубоко связанными данными. Они также позволяют легко добавлять новые данные в граф без необходимости изменения схемы базы данных.
Однако, реляционные базы данных обладают более высокой надежностью и могут обеспечивать лучшую производительность при выполнении сложных запросов, особенно если используются правильно настроенные индексы.

Таким образом, выбор между реляционными базами данных и графовыми зависит от конкретных требований проекта.
В контексте данной работы, графовая модель подходит больше, чем реляционная, потому что тест Тьюринга включает в себя множество связей между объектами (человек, машина, эксперт, вопросы и ответы и т.д.), которые можно представить в виде графа.

Графовая модель также становится еще более привлекательной, если вспомнить какая идея была обозначена в начале данной работы: создание инструмента для улучшения искусственного интеллекта. В большинстве случаев ИИ работает не со стандартными <<табличными>> данными, а данными, представленных в виде графа. Таким образом, схожая структура данных внутри СУБД поможет разработать более гибкую и быстродействующую систему при меньших затраченных ресурсах

На рис.~\ref{img:graph-bench} можно увидеть результаты сравнения 3 различных баз данных: реляционной (PostgreSQL), графовой (Neo4j) и мультимодельной (ArangoDB).

\img{100mm}
{graph-bench}
{Сравнение времени работы различных баз данных над атомарными операциями.}

\clearpage

\subsection{Системы управления базами данных}

Для выбора системы управления базами данных необходимо учитывать требования к производительности и масштабируемости приложения.

Реляционные базы данных имеют высокую надежность и поддерживают ACID-свойства транзакций. Они также обеспечивают хорошую производительность при выполнении сложных запросов. Однако, они требуют дополнительного управления индексами и ключевыми полями.

Графовые базы данных обеспечивают высокую производительность при работе с глубоко связанными данными. Граф связывает элементы данных в хранилище с набором узлов и ребер, причем сами ребра представляют отношения между узлами. Отношения позволяют напрямую связывать данные в хранилище и во многих случаях извлекать их с помощью одной операции. Отношения между данными в подобных системах имеют приоритет над самими данными, поэтому запрос по отношениям является крайне быстрой операцией, поскольку они постоянно хранятся в базе данных. Также отношения можно интуитивно визуализировать с помощью баз данных графов, что делает их полезными для сильно взаимосвязанных данных. 

\subsubsection{SurrealDB}

SurrealDB --- мультимодельная NewSQL база данных, которая работает в режиме полной схемы или без схемы, с таблицами, ссылками на записи между документами (без JOIN) и функциями моделирования базы данных на основе графов.

Благодаря использованию SurrealDB особых методов сегментирования и репликации, становится возможным повысить производительность за счет распределения нагрузки между несколькими компьютерами.

Также особая архитектура базы данных позволяет работать как in-memory, on-disk или как распределенная база данных, используя TiKV.

Поскольку SurrealDB --- мультимодельная база данных, становится также возможным классические реляционные методики проектирования баз данных, что повышает гибкость итоговой системы.

\subsubsection{Neo4j}

Neo4j --- это графовая база данных, которая позволяет хранить, управлять и анализировать связанные данные. Она была разработана с учетом графовой модели данных, в которой данные представлены в виде узлов (вершин) и связей (ребер). 

Одним из преимуществ Neo4j является то, что она позволяет эффективно моделировать и анализировать сложные связи между данными, такие как социальные сети, географические карты и сети предприятий. Это делает ее очень полезной для приложений, которые требуют быстрого доступа к сложным данным и быстрой обработки запросов.

Однако поскольку Neo4j ---  исключительно графовая база данных, хранение и получение данных без каких-либо связей друг с другом может вызвать проблемы с производительностью, вне зависимости от размера запроса.

\subsection{Выбор СУБД для решения задачи}

Для решения задачи теста Тьюринга необходимо выбрать графовую базу данных, поскольку графовая модель данных более естественным образом отображает связи между объектами. 

Среди графовых баз данных можно выделить две наиболее подходящие системы: SurrealDB и Neo4j. Обе СУБД обеспечивают быстрый доступ к данным по отношениям, что делает их эффективными при работе с глубоко связанными данными. 

Однако SurrealDB имеет дополнительные преимущества перед Neo4j. Она является мультимодельной базой данных, что позволяет эффективное хранение и получение несвязанных данных, где Neo4j может испытывать определенные проблемы.

\subsection{Проблемы кэширования данных и их решение}

Одной из проблем, связанных с хранением и обработкой данных в базах данных, является производительность. Возможны ситуации, когда приложение должно быстро получать данные из базы данных, но поиск этих данных может занять много времени. Кэширование может помочь решить эту проблему.

Кэширование --- это метод хранения данных в памяти для быстрого доступа к ним. Кэш представляет собой временное хранилище данных, которые часто запрашиваются из базы данных. Если данные уже находятся в кэше, приложение может получить доступ к ним намного быстрее, чем если они были запрошены непосредственно из базы данных.

Для кэширования данных можно использовать NoSQL in-memory базы данных. Такие базы данных хранят данные в оперативной памяти, что обеспечивает более быстрый доступ к данным. Примерами таких баз данных могут считаться Redis и Tarantool.
Однако использование кэширования может столкнуться со следующими проблемами:
\begin{enumerate}
  \item Неконсистентность данных. Если данные изменились в базе данных, но не обновлены в кэше, то будет получен неверный результат. Решением этой проблемы является использование механизма инвалидации кэша, который обновляет данные в кэше при изменении в базе данных.
  \item Ограничение объема памяти. Кэш может занять значительное количество оперативной памяти, что может привести к ограничениям по объему хранимых данных. Решением этой проблемы может быть использование LRU-алгоритма (Least Recently Used), который удаляет менее используемые данные из кэша для освобождения памяти.
  \item Потеря данных при сбое системы. Если происходит сбой системы или выключение сервера, то данные в кэше могут быть потеряны. Решением этой проблемы может быть использование репликации и бекапов для сохранения данных.
  \item Производительность записи в базу данных. Использование кэша может повысить производительность чтения данных, но при записи данных в базу производительность может снижаться из-за необходимости обновления как в базе данных, так и в кэше. Решением этой проблемы является оптимизация процесса записи данных и использование асинхронных запросов.
\end{enumerate}

\subsection{Обзор in-memory NoSQL СУБД}

\subsubsection{Tarantool}

Tarantool \cite{tarantool} --- это платформа in-memory вычислений с гибкой схемой хранения данных для эффективного создания высоконагруженных приложений. Включает себя базу данных и сервер приложений на языке программирования Lua \cite{lua}.

Tarantool обладает высокой скоростью работы по сравнению с традиционными СУБД. При этом, в рассматриваемой платформе для транзакций реализованы свойства ACID, репликация master-slave \cite{master-slave} и master-master \cite{master-master}, как и в традиционных СУБД.

Для хранения данных используется кортежи (англ. tuple) данных. Кортеж -- это массив не типизированных данных. Кортежи объединяются в спейсы (англ. space), аналоги таблицы из реляционной модели хранения данных. Спейс -- коллекция кортежей, кортеж -- коллекция полей.

В Tarantool реализован механизм <<снимков>> текущего состояния хранилища и журналирования всех операций, что позволяет восстановить состояние базы данных после ее перезагрузки.\\

\subsubsection{Redis}
Redis \cite{redis} --- резидентная система управлениями базами данных класса NoSQL с открытым исходным кодом. Основной структурой данных, с которой работает Redis является структура типа <<ключ-значение>>. Данная СУБД используется как для хранения данных, так и для реализации кэшей и брокеров сообщений.

Redis хранит данные в оперативной памяти и снабжена механизмом <<снимков>> и журналирования, что обеспечивает постоянное хранение данных. Предоставляются операции для реализации механизма обмена сообщениями в шаблоне <<издатель-подписчик>>: с его помощью приложения могут создавать программные каналы, подписываться на них и помещать в эти каналы сообщения, которые будут получены всеми подписчиками. Существует поддержка репликации данных типа master-slave, транзакций и пакетной обработки команд.

Все данные Redis хранит в виде словаря, в котором ключи связаны со своими значениями. Ключевое отличие Redis от других хранилищ данных заключается в том, что значения этих ключей не ограничиваются строками. Поддерживаются следующие абстрактные типы данных:

\begin{itemize}
\item строки;
\item списки;
\item множества;
\item хеш-таблицы;
\item упорядоченные множества.
\end{itemize}

Тип данных значения определяет, какие операции доступные для него; поддерживаются высокоуровневые операции: например, объединение, разность или сортировка наборов.

\subsection{Выбор СУБД для кэширования данных}

Для кэширования данных можно использовать любую in-memory NoSQL базу данных. Обе системы, Tarantool и Redis, обеспечивают быстрый доступ к данным в памяти и имеют механизмы инвалидации кэша. 

Однако, Tarantool обладает дополнительными преимуществами перед Redis. Она предоставляет возможность хранения данных в гибком формате кортежей, что позволяет более эффективно организовывать данные в базе данных. Также в Tarantool реализованы свойства ACID для транзакций и механизм репликации данных, что делает ее более надежной для использования в высоконагруженных приложениях.

Поэтому для решения задачи кэширования данных лучше выбрать Tarantool.

\subsection*{Вывод}

В данном разделе были рассмотрены различные типы баз данных: реляционные и графовые, а также обзор in-memory NoSQL СУБД. Было выявлено, что для решения задачи теста Тьюринга наиболее подходящей является графовая база данных, а конкретно SurrealDB. Для кэширования данных можно использовать любую in-memory NoSQL базу данных, но наилучшим выбором для этой задачи является Tarantool.








% При работе с базами данных возникает проблема кэширования данных. Кэш - это временное хранилище для часто запрашиваемых данных, которые сохраняются в оперативной памяти для быстрого доступа.
%
% Одна из проблем кэширования данных - это несогласованность данных. Если данные в базе данных изменены, а кэш не обновлен, то приложение может использовать устаревшие данные.
%
% Другая проблема - это проблема выборки (cache miss). Если запрашиваемые данные не находятся в кэше, то приложение должно выполнить запрос к базе данных, что может замедлить работу приложения.
%
% \section{Методы решения проблем кэширования данных}
%
% Для решения проблем кэширования данных можно использовать различные подходы:
% \begin{itemize}
%     \item[$-$] Использование инвалидации кэша. При изменении данных в базе данных, кэш помечается как недействительный (invalidated), и приложение обновляет данные при следующем запросе.
%     \item[$-$] Использование временных меток (timestamps). При каждом изменении данных в базе данных, записывается временная метка, которая позволяет определить, являются ли данные в кэше устаревшими.
%     \item[$-$] Использование стратегий вытеснения (cache eviction). Если кэш заполнен, то необходимо выбрать данные для удаления из кэша. Существуют различные стратегиивытеснения, включая:
% \end{itemize}
%
% \begin{enumerate}
%     \item Наименее используемые (Least Recently Used, LRU) - удаляет элементы, которые были меньше всего запрашиваемы за последнее время.
%     \item Наименее часто используемые (Least Frequently Used, LFU) - удаляет элементы, которые были запрашиваемы наименее количество раз.
%     \item Случайные (Random) - случайным образом выбирает элементы для удаления.
%     \item По времени жизни (Time to Live, TTL) - удаляет элементы, которые превысили заданный период жизни в кэше.
% \end{enumerate}
%
% Выбор конкретной стратегии зависит от требований и характеристик приложения, а также от доступных ресурсов и ограничений. 

\section{Конструкторский раздел}
В данном разделе представлены этапы проектирования выделенной в предыдущем разделе базы данных, нужной для решения задачи.

\subsection{Проектирование базы данных для хранения Тестов Тьюринга}  
База данных для хранения Тестов Тьюринга будет реализована с использованием СУБД SurrealDB.
В базе данных будет существовать 7 таблиц и 7 типов отношений. 
Схема разработанной базы данных представлена на рисунке~\ref{img:er_new}.

\img{125mm}
{er_new}
{Схема разработанной базы данных.}

Поля таблицы \texttt{Interrogator}: 
\begin{enumerate}
    \item \texttt{UniqueID} --- уникальный идентификатор --- \texttt{UUID};
    \item \texttt{Name} --- имя <<эксперта>> --- строка;
    \item \texttt{Gender} --- пол <<эксперта>> --- строка;
    \item \texttt{Age} --- возраст <<эксперта>> --- целое число;
    \item \texttt{Nationality} --- национальность <<эксперта>> --- строка.
\end{enumerate}

Данная таблица отвечает за хранение данных, связанных с экспертом, проводящим эксперимент.
Эта таблица связана со следующими таблицами:
\begin{itemize}
    \item[$-$] \texttt{Session} --- через отношение \texttt{:ParticipateIn};
    \item[$-$] \texttt{Answer} --- через отношение \texttt{:Asks};
    \item[$-$] \texttt{Verdict} --- через отношение \texttt{:Gives}.
\end{itemize}

Поля таблицы \texttt{Computer}:
\begin{enumerate}
    \item \texttt{UniqueID} --- уникальный идентификатор --- \texttt{UUID};
    \item \texttt{Model} --- Модель ИИ, проходившая тест --- строка;
    \item \texttt{DevelopedBy} --- Разработчики указанной модели ИИ --- строка.
\end{enumerate}

Данная таблица отвечает за хранение данных, связанных с компьютером, участвующем в эксперименте. 
Эта таблица связана со следующими таблицами:
\begin{itemize}
    \item[$-$] \texttt{Session} --- через отношение \texttt{:ParticipateIn};
    \item[$-$] \texttt{Answer} --- через отношение \texttt{:Gives};
    \item[$-$] \texttt{Verdict}. --- через отношение \texttt{:Mentions}.
\end{itemize}

Поля таблицы \texttt{Human}:
\begin{enumerate}
    \item \texttt{UniqueID} --- уникальный идентификатор --- \texttt{UUID};  
    \item \texttt{Name} --- имя человека --- строка;
    \item \texttt{Gender} --- пол человека --- строка;
    \item \texttt{Age} --- возраст человека --- строка; 
    \item \texttt{Nationality} --- национальность человека --- строка.
\end{enumerate}

Данная таблица отвечает за хранение данных, связанных с человеком, участвующем в эксперименте. 
Эта таблица связана со следующими таблицами:
\begin{itemize}
    \item[$-$] \texttt{Session} --- через отношение \texttt{:ParticipateIn};
    \item[$-$] \texttt{Answer} --- через отношение \texttt{:Gives};
    \item[$-$] \texttt{Verdict} --- через отношение \texttt{:Mentions}.
\end{itemize}

Поля таблицы \texttt{Answer}:
\begin{enumerate}
    \item \texttt{ItemID} --- уникальный идентификатор --- \texttt{UUID}.
    \item \texttt{AnswerText} --- текст ответа --- строка.
\end{enumerate}

Данная таблица отвечает за хранение данных, связанных с ответами, данными в эксперименте. 
Следует отметить, что ответы, данные на протяжении всех экспериментов, являются уникальными сущностями, или, иными словами, в базе данных нет двух одинаковых ответов на любой из вопросов.
Данная особенность преследует цель показать связь между вопросами и ответами, и как различные вопросы могут привести к одним и тем же ответам, либо же, как компьютер и человек в экперименте могут дать одинаковый ответ.

Эта таблица связана со следующими таблицами:
\begin{itemize}
    \item[$-$] \texttt{Session} --- через отношение \texttt{:Includes};
    \item[$-$] \texttt{Answer} --- через отношения \texttt{:AnsweredBy} и \texttt{:Follows};
    \item[$-$] \texttt{Computer} --- через отношение \texttt{:Gives};
    \item[$-$] \texttt{Human} --- через отношение \texttt{:Gives};
    \item[$-$] \texttt{Verdict} --- через отношение \texttt{:Follows}.
\end{itemize}

Поля таблицы \texttt{Question}:
\begin{enumerate}
    \item \texttt{ItemID} --- уникальный идентификатор --- \texttt{UUID}.
    \item \texttt{QuestionText} --- текст вопроса --- строка.
\end{enumerate}

Данная таблица отвечает за хранение данных, связанных с вопросами, данными в эксперименте экспертом. 
Следует отметить, что вопросы, данные на протяжении всех экспериментов, как и ответы, упомянутые выше, являются уникальными сущностями, или, иными словами, в базе данных нет двух одинаковых вопросов.
Данная особенность преследует цель показать связь между вопросами и ответами, и как на один вопрос можно привести множество различных ответов.

Эта таблица связана со следующими таблицами:
\begin{itemize}
    \item[$-$] \texttt{Session} --- через отношение \texttt{:Includes};
    \item[$-$] \texttt{Answer} --- через отношения \texttt{:AnsweredBy} и \texttt{:Follows};
    \item[$-$] \texttt{Interrogator} --- через отношение \texttt{:Asks}.
\end{itemize}

Поля таблицы \texttt{Verdict}:
\begin{enumerate}
    \item \texttt{ItemID} --- уникальный идентификатор --- \texttt{UUID};
    \item \texttt{Correct} --- Верен ли вердикт, выданный экспертом --- ложь / истина.
\end{enumerate}

Данная таблица отвечает за хранение данных, связанных с вердиктами, данными экспертами по окончанию экспериментов.
После любого данного ответа, эксперт может закончить эксперимент и выдать свой вердикт, кто является компьютером, а кто человеком.

Эта таблица связана со следующими таблицами:
\begin{itemize}
    \item[$-$] \texttt{Session} --- через отношение \texttt{:Includes};
    \item[$-$] \texttt{Answer} --- через отношение \texttt{:Follows};
    \item[$-$] \texttt{Interrogator} --- через отношение \texttt{:Gives};
    \item[$-$] \texttt{Computer} --- через отношение \texttt{:Mentions};
    \item[$-$] \texttt{Human} --- через отношение \texttt{:Mentions}.
\end{itemize}

Поля таблицы \texttt{Session}:
\begin{enumerate}
    \item \texttt{SessionID} --- уникальный идентификатор --- \texttt{UUID};
    \item \texttt{TimeFrame} --- период времени, отведенный на ответ на вопрос --- время;
    \item \texttt{TimeSpent} --- продолжительность сессии --- время.
\end{enumerate}

Данная таблица отвечает за хранение данных, связанных с различными экспериментами.
Данная таблица является свеого рода мета-таблицей, по связи с которой можно получить данные о всех сущностях, участвующих в эксперименте.

Эта таблица связана со следующими таблицами:
\begin{itemize}
    \item[$-$] \texttt{Question} --- через отношение \texttt{:Includes};
    \item[$-$] \texttt{Answer} --- через отношение \texttt{:Includes};
    \item[$-$] \texttt{Answer} --- через отношение \texttt{:Includes};
    \item[$-$] \texttt{Interrogator} --- через отношение \texttt{:ParticipateIn};
    \item[$-$] \texttt{Computer} --- через отношение \texttt{:ParticipateIn};
    \item[$-$] \texttt{Human} --- через отношение \texttt{:ParticipateIn}.
\end{itemize}

Поля рёбер \texttt{:AnsweredBy} и \texttt{:Follows}:
\begin{enumerate}
    \item \texttt{Order} --- порядковый номер вопроса/ответа/вердикта в системе - целое число
\end{enumerate}

% Пример SQL запроса, утилизирующий множественные JOIN-операции: 

Особенность \texttt{SurrealDB} заключается в том, что отношения также могут иметь дополнительные поля, характеризующие их.
Поле \texttt{Order} необходимо для построения контекста ответов/вопросов поскольку ответ может различаться от того, какие ответы были даны ранее.

\subsection{Структура разрабатываемого ПО}
Предполагается, что разрабатываемый проект является одним цельным \texttt{Electron}-подобным~\cite{electron} приложением.
Серверная часть и графический интерфейс упаковывается в единый бинарный файл, предоставляя возможность создать различные версии приложения под различные операционные системы.
В общем смысле, серверная часть коммуницирует с базой данных, доставляя результат к графическому интерфейсу в рамках единого приложения.

Общая схема архитектура приложения представлена на рисунке~\ref{img:arch}

\img{75mm}
{arch}
{Схема архитектуры приложения}


\subsection*{Вывод}
В данном разделе были представлены проектирование базы данных, рассмотрены особенности используемой СУБД на архитектурном уровне и была показана структура разрабатываемого ПО.

% \section{Технологический раздел}

\subsection{Средства реализации}

Основным языком программирования является мультипарадигменный язык Rust~\cite{rust}.
\begin{itemize}
    \item[$-$] Одно из главных достоинств данного языка это гарантия безопасной работы с памятью при помощи системы
            статической проверки ссылок, так называемый \texttt{Borrow Checker}~\cite{borrow-checker}.
    \item[$-$] Отсутствие сборщика мусора, как следствие, более экономная работа с ресурсами.
    \item[$-$] Встроенный компилятор, постaвляемый совместно с пакетным менеджером \texttt{Cargo}.
    \item[$-$] Кросс-платформенность, от \texttt{UNIX} и \texttt{MacOS} приложений до Web - приложений.
    \item[$-$] \texttt{SurrealDB} написан на языке \texttt{Rust}, в следствии чего инструментарий языка наиболее плотно работает с непосредственно самой базой данных.
    \item[$-$] Важно отметить, что язык программирования \texttt{Rust} сопоставим по скорости с такими языками как \texttt{С}~\cite{c-rust} и \texttt{С++}~\cite{cpp-rust},
    предоставляя в то же время более широкий функционал для тестирования кода и контроля памяти.
\end{itemize}

Также в рамках языка \texttt{Rust} был выбран фреймворк \texttt{Tauri}. 
\texttt{Tauri} используется для создания приложений с использованием комбинации инструментов \texttt{Rust} и \texttt{HTML}, отображаемых в \texttt{Webview}. 
Приложения, созданные с помощью \texttt{Tauri}, могут поставляться с любым количеством дополнительных \texttt{JS API} и \texttt{Rust API}, так что \texttt{Webview} может управлять системой посредством передачи сообщений. 
Разработчики могут расширить API за счет своей собственной функциональности и легко объединить \texttt{Webview} и серверную часть на основе \texttt{Rust}~\cite{tauri}.
На рисунке~\ref{img:tauri_new} изображена общая архитектура \texttt{Tauri}-приложения.

\img{100mm}
{tauri_new}
{Принцип работы \texttt{Tauri}}

\subsection{Детали реализации}
На листинге~\ref{lst:rep} представлены трейты, аналог интерфейса в языке \texttt{Rust}, необходимые для реализации репозитория, который отвечает за взаимодействие между публичным \texttt{API} и непосредственно с базой данных.

\begin{lstinputlisting}[
        caption={\raggedrightТрейты, необходимые для реализации репозитория.},
        label={lst:rep},
        language={rust},
        style={rustlang}
        % linerange={1-58}
    ]{../../src-tauri/src/repository/mod.rs}
\end{lstinputlisting}

На листинге~\ref{lst:rep-impl} представлена реализация репозитория и моделей, необходимых для трансляции данных из \texttt{SurrealDB} в пространство языка \texttt{Rust}.
\begin{lstinputlisting}[
        caption={\raggedrightРеализация репозитория.},
        label={lst:rep-impl},
        style={rustlang},
        language={rust},
        linerange={81-225}
    ]{../../src-tauri/src/repository/surrealdb.rs}
\end{lstinputlisting}

На листинге~\ref{lst:init} представлены запросы, которые вызываются при инициализации базы данных.
\begin{lstinputlisting}[
        caption={\raggedrightИнициализация базы данных.},
        label={lst:init},
        language={rust},
        style={rustlang}
        % linerange={81-225}
    ]{../../src-tauri/build/init.sql}
\end{lstinputlisting}

На листинге~\ref{lst:api} представлены 3 метода API, которые вызываются со стороны графического интерфейса.
\begin{lstinputlisting}[
        caption={\raggedrightМетоды API.},
        label={lst:api},
        language={rust},
        style={rustlang},
        linerange={16-100}
    ]{../../src-tauri/src/api.rs}
\end{lstinputlisting}

\subsection{Интерфейс программы}
Для работы с БД был разработан графический интерфейс.
Для реализации интерфейса был использован фреймворк \texttt{Next.js} с использованием языка \texttt{Typescript}.
В программного интерфейсе реализованы простейшие операции, связанные с просмотром данных об эксперимента. 
Также добавлены функции входа - выхода из системы.
Графический интерфейс представлен на рисунках~\ref{img:face},~\ref{img:face2},~\ref{img:face3},~\ref{img:face4} и~\ref{img:face5}.

\img{100mm}
{face}
{Страница регистрации нового пользователя}

\img{100mm}
{face2}
{Страница входа в систему}

\img{100mm}
{face5}
{Страница с информацией о пользователе}

\img{100mm}
{face3}
{Страница с доступными экспериментами}

\img{100mm}
{face4}
{Страница с экспериментом}

\clearpage

\subsection*{Вывод}
В данном разделе были представлена средства реализации программного обеспечения, листинги ключевых компонентов системы и пример работы приложения.

% Среда разработки:

%
% В процессе разработки был использован инструмент LSP\cite{lsp} (англ. \textit{Language Server Protocol}), а в частности его реализацию в виде Rust Analyzer~\cite{rust-anal}, позволяющий форматировать исходные коды, а также в процессе их написания обнаружить наличие синтаксических ошибок и некоторых логических, таких как, например, нарушение правила владения\cite{rust-learn}.
%
% В качестве среды разработки был выбран текстовый редактор VIM\cite{vim}, поддерживающий возможность установки плагинов\cite{vim-plugins}, в том числе для работы с LSP\cite{lsp}.
% \subsection{Структура классов}
%
% На рисунках \ref{img:classes_A} - \ref{img:classes_C} представлена структура реализуемых классов.
%
% \img{100mm}
% {classes_A} % Имя файла без расширения (файл должен быть расположен в директории inc/img/)
% {Структура классов-объектов} % Подпись рисунка
%
% \begin{itemize}
%     \item Point – класс точки трехмерного пространства. Хранит координаты в пространстве, владеет методами преобразований точки.
%     \item Edge – класс грани. Хранит номера задействованных в грани вершин.
%     \item Light – класс источника света.
%     \item Model - класс модели. Скрывает конкретную реализацию модели(фигуры) и предоставляет единый интерфейс для работы с ней. Владеет методами преобразования модели, а также методами для получения информации о модели.
%     \item Composite - класс композита. Хранит в себе набор моделей, владеет методами для работы с ними.
% \end{itemize}
%
% \img{100mm}
% {classes_C} % Имя файла без расширения (файл должен быть расположен в директории inc/img/)
% {Структура классов} % Подпись рисунка
%
% \begin{itemize}
%     \item Drawer – класс, отвечающий за растеризацию сцены. Хранит полотно для отрисовки. Владеет методами алгоритма теневого z-буфера и формирования объекта для отображения рисунка в главном приложении.
%     \item App – точка входа в программу.
%     \item Ui - класс, отвечающий за отображение графического интерфейса.
%     \item TransformManager – абстракция, содержащия методы трансформмации объектов.
%     \item LoadManager - абстракция, содержащия методы загрузки объектов.
%     \item Canvas - класс, отвечающий за отображение сцены.
% \end{itemize}
% \subsection{Реализация алгоритмов}
% В листинге \ref{lst:cg} представлена реализация Z-буффера и Гуро на языке Rust.
% В листинге \ref{lst:frame_model} представлена реализация 3D модели.
%
% \begin{lstinputlisting}[
%         caption={Реализация алгоритмов компьютерной графики.},
%         label={lst:cg},
%         style={rustlang},
%         linerange={33-262}
%     ]{../../../src/app_factory/drawer/drawer_std.rs}
% \end{lstinputlisting}
%
% \begin{lstinputlisting}[
%         caption={Реализация 3D модели.},
%         label={lst:frame_model},
%         style={rustlang},
%         linerange={24-290}
%     ]{../../../src/models/frame_model.rs}
% \end{lstinputlisting}
%
% \subsection{Интерфейс программного обеспечения}
%
% При запуске программы перед пользователем предстает пустая сцена. 
% Для операций над объектами, их модифицирования, для управления камерой или освещением в левой части интерфейса определены соответствующие разделы (рисунок~\ref{img:example}).
%
% \img{100mm}{example}{GUI}
%
% Для создания модели пользователю необходимо нажать на соответствующий раздел и выбрать одно из предложенных тел. Далее в разделе моделей пользоватеь может каких-либо образом изменить необходимую модель. (рисунки ~\ref{img:create}-\ref{img:modify}).
%
% \clearpage
%
% \img{100mm}{create}{Раздел создание модели}
%
% \img{100mm}{modify}{Раздел модифицирования модели}
% Для перемещения по сцене используются клавиши: W – вперед, S – назад, A – влево, D – вправо, Стрелка вверх - вверх, Стрелка вниз - вниз, R и T - вращение камеры, Z и С - поворот камеры. 
% % \clearpage
%
% В разделе камеры находятся следующие параметры: расстояния до ближней и дальней плоскостей пирамиды видимости, угол обзора, позиция камеры в пространстве, углы поворота, скорость перемещения (рисунок~\ref{img:camera}).
%
% \img{100mm}{camera}{Раздел камеры}
%
% \clearpage
%
% На рисунке~\ref{img:example-models} приведен пример работы программы.
%
% \img{100mm}{example-models}{Тестовая сцена}
% \subsection*{Вывод}
%
% В данном разделе были рассмотрены средства, с помощью которых было реализовано ПО, а также представлены структуры классов и листинги кода с реализацией алгоритмов компьютерной графики.

% \section{Исследовательский раздел}

В данном разделе приведены описание исследования и технические характеристики устройства, на котором проводилось измерение времени работы программного обеспечения, а также результаты замеров времени.

\subsection{Постановка задачи исследования}

\subsubsection{Цель исследования}

Целью исследования является сравнение времени, требуемого для получения сильносвязанных данных о тесте Тьюринга в двух базах данных \texttt{SurrealDB} и \texttt{PostgresQL}.

\subsubsection{Описание исследования}
Cравнить занимаемое время для получения данных для различных баз данных можно при помощи бенчмарков --- специальных функций, котороые проводят серии различных испытыний с записью производительности системы для дальнейшего их сравнения.
Для \texttt{SurrealDB} в рамках бенчмаркинга были написаны запросы, утилизирующие её графовую составляющую.
В то же время запросы к \texttt{PostgreSQL} применяют множественные JOIN-запросы ввиду сильносвязанности данных.

На листинге~\ref{lst:postgres} приведён SQL-запрос для \texttt{PostgreSQL}, по которому проводилось сравнение. 
Следует отметить, что в случае PostgresQL связь между таблицами осуществляется при помощи внешних ключей (\texttt{Foreign Key}). 
Во всем остальном, таблицы аналогичны тем, что написаны для \texttt{SurrealDB}.

\clearpage

\begin{lstlisting}[
		caption={SQL запрос для \texttt{PostgreSQL}.},
		label={lst:postgres}
	]
    SELECT 
    		Session.SessionID,
        Interrogator.Name, 
        Computer.Model, 
        Human.Name, 
        Question.QuestionText,
        Answer.AnswerText,
        Verdict.Correct
    FROM Session
        JOIN Interrogator ON Session.SessionID = Interrogator.SessionID
        JOIN Computer ON Session.SessionID = Computer.SessionID
        JOIN Human ON Session.SessionID = Human.SessionID
        JOIN Question ON Session.SessionID = Question.SessionID
        JOIN Answer ON Session.SessionID = Answer.SessionID
        JOIN Verdict ON Session.SessionID = Verdict.SessionID
\end{lstlisting}

Аналогичный запрос написанный для \texttt{SurrealDB}, утилизирующий графовую составляющую базы данных:

\begin{lstlisting}
SELECT id,
			<-participateIn<-human.name,
			<-participateIn<-interroagtor.name, 
			<-participateIn<-computer.model, 
			->includes->answer.text, 
			->includes->question.text, 
			->includes->verdict.correct 
						FROM session;
\end{lstlisting}

Для замера производительности двух ралзичных баз данных при выполнении запросов использовались библиотека \texttt{Criterion}, функции которой использовались для определения эффективности запросов по времени.
Количество итераций и алгоритм анализа выборок контролируются библиотекой~\cite{criterion-anal} и работают следующим образом:
\begin{itemize}
	\item[$-$] Перед началом самих замеров проводится этап <<разогрева>>. 
	На этом этапе процедура выполняется повторно, чтобы дать операционной системе и процессору время адаптироваться к новой рабочей нагрузке.
	\item[$-$] Измерения проводятся на нескольких выборках. Каждая выборка состоит из одной или нескольких итераций процедуры.

	По мере выполнения измерений количество итераций выборки увеличивается. Количество итераций вычисляется следующим образом:

	\texttt{Количество итераций = [d, 2d, 3d, ... Nd]}

	Где N - общее количество выборок, а d - коэффициент, рассчитанный на основе приблизительной оценки времени итерации, измеренной в течение периода разогрева, который используется для масштабирования количества итераций в соответствии с настроенным временем измерения --- 60 секунд.
	
	\item[$-$] Первым шагом в анализе является классификация выбросов. 
		Каждая выборка классифицируется с использованием модифицированной версии метода Тьюки~\cite{tukey}.
		Во-первых, межквартильный диапазон (\texttt{IQR}) рассчитывается исходя из разницы между 25 - м и 75 - м процентилями. 
		В методе Тьюки значения меньше (25 - й процентиль - 1,5 * IQR) или больше (75 - й процентиля + 1,5 * IQR) считаются выбросами. 

		На втором шаге среди выборок вычисляется ряд метрик. 
		В данном исследовании используется среднее значение полученных выборок.
\end{itemize}

\subsubsection{Технические характеристики}

Ниже приведены технические характеристики устройства, на котором будет проведено исследование:

\begin{itemize}
    \item[$-$] Операционная система: Arch Linux~\cite{arch-linux} 64-bit;
		\item[$-$] Количество ядре: 4 физических и 8 логических ядер;
    \item[$-$] Оперативная память: 16 Гб, DDR4;
    \item[$-$] Процессор: 11th Gen Intel\textsuperscript{\tiny\textregistered} Core\textsuperscript{\tiny\texttrademark} i5-11320H @ 3.20 ГГц~\cite{i5}.
\end{itemize}

Во время тестирования устройство было подключено к сети электропитания и было нагружено только встроенными приложениями окружения, а также непосредственно системой тестирования.

\subsubsection{Результаты исследования}

В таблицах \ref{tbl:experiment1} - \ref{tbl:experiment3} представлены результаты поставленного эксперимента, где сравнивается время исполнения в зависимости от количесвта сущностей в базе данных.

\clearpage
\begin{table}[ht!]
	\centering
	\caption{Результаты сравнения времени, для запросов к \texttt{PostgreSQL} и \texttt{SurrealDB} (количество элементов - 100 единиц)}
	\label{tbl:experiment1}
	\resizebox{\textwidth}{!}{%
		\begin{tabular}{|l|l|l|}
			\hline
			\textbf{\begin{tabular}[c]{@{}c@{}}Количество \\ запросов\end{tabular}} & \textbf{\texttt{SurrealDB}, мс} & \textbf{\texttt{PostgresQL}, мс}  \\ \hline
			1 & 58732 & 45812 \\ \hline
			5 & 255079 & 224391 \\ \hline
			10 & 527629 & 473282 \\ \hline
			25 & 1428489 & 1262168 \\ \hline
			100 & 5021948 & 4624102 \\ \hline
		\end{tabular}%
	}
\end{table}

\begin{table}[ht!]
	\centering
	\caption{Результаты сравнения времени, для запросов к \texttt{PostgreSQL} и \texttt{SurrealDB} (количество элементов - 1000 единиц)}
	\label{tbl:experiment2}
	\resizebox{\textwidth}{!}{%
		\begin{tabular}{|l|l|l|}
			\hline
			\textbf{\begin{tabular}[c]{@{}c@{}}Количество \\ запросов\end{tabular}} & \textbf{\texttt{SurrealDB}, мс} & \textbf{\texttt{PostgresQL}, мс}  \\ \hline
			1 & 63418 & 68719 \\ \hline
			5 & 284249 & 301548 \\ \hline
			10 & 577324 & 598925  \\ \hline
			25 & 1531132 & 1762836 \\ \hline
			100 & 6145253 & 7843628 \\ \hline
		\end{tabular}%
	}
\end{table}

\begin{table}[ht!]
	\centering
	\caption{Результаты сравнения времени, для запросов к \texttt{PostgreSQL} и \texttt{SurrealDB} (количество элементов - 5000 единиц)}
	\label{tbl:experiment3}
	\resizebox{\textwidth}{!}{%
		\begin{tabular}{|l|l|l|}
			\hline
			\textbf{\begin{tabular}[c]{@{}c@{}}Количество \\ запросов\end{tabular}} & \textbf{\texttt{SurrealDB}, мс} & \textbf{\texttt{PostgresQL}, мс} \\ \hline
			1 & 76418 & 95213 \\ \hline
			5 & 328253 & 414436 \\ \hline
			10 & 663635 & 737435  \\ \hline
			25 & 1931132 & 2435168 \\ \hline
			100 & 7296236 & 9396236 \\ \hline
		\end{tabular}%
	}
\end{table}

\clearpage

На рисунках \ref{img:plot1} - \ref{img:plot3} представлены визулизация результатов поставленного эксперимента в виде графиков.

\img{100mm}
{plot1}
{Зависимость времени от количества запросов (количество элементов --- 100)}

\img{100mm}
{plot2}
{Зависимость времени от количества запросов (количество элементов --- 1000)}

\img{100mm}
{plot3}
{Зависимость времени от количества запросов (количество элементов --- 5000)}

\clearpage

\subsection*{Вывод}
Результаты эксперимента показали, что, в то время как при небольшом количестве сущностей в базе данных \texttt{PostgresQL} показывает лучшие результаты, при увеличении количества сущностей в базе данных \texttt{SurrealDB} показывает лучшие результаты.
Из данного наблюдения можно сделать следующий вывод:
графовые базы данных показывают лучшие в сравнении с реляционными базами данных результаты при большом количестве сущностей в базе данных.


% \specsection{ЗАКЛЮЧЕНИЕ}
% \addcontentsline{toc}{specsection}{ЗАКЛЮЧЕНИЕ}
В ходе выполнения проекта, цель данного курсовой работы была достигнута, то есть был разработан программный продукт, позволяющий эффективно хранить результаты тестов Тьюринга.
В ходе выполнения экспериментально-исследовательской части было установлено, что запросы по связям будут выполняться быстрее в графовых базах данных, нежели запросы с использованием множественных JOIN-запросов в реляционных базах данных.
При этом следует помнить, что значительные выигрыши по времени могут наблюдаться только при условии большого размера базы данных и сильной связанности сущностей внутри неё.

Для достижение цели были выполнены следующие задачи:
\begin{itemize}
    \item[$-$] проведен анализ предметной области --- Тест Тьюринга;
    \item[$-$] спроектирована архитектура программного обеспечения, на которой были отображены сущности и возможные связи между ними;
    \item[$-$] выбраны средства реализации программного обеспечения c подробным описанием их особенностей;
    \item[$-$] разработано программное обеспечение;
    \item[$-$] разработаны и проведены эксперименты по замеру времени работы программного обеспечения.
\end{itemize}




\include{09-biblio}
% \section*{\centering ПРИЛОЖЕНИЕ А}
% \addcontentsline{toc}{section}{ПРИЛОЖЕНИЕ А}
%
% \section*{Презентация к курсовой работе}
% Презентация содержит 13 слайдов.
\end{document}
