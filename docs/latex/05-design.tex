\section{Конструкторский раздел}
В данном разделе представлены этапы проектирования выделенных в  
предыдущем разделе баз данных, нужных для решения задачи

\subsection{Проектирование отношений сущностей}  

На рисунке представлена схема сущностей, необходимых для реализации конечного приложения

\img{100mm}
{tarantool}
{Схема сущностей приложения.}

\subsection{Проектирование базы данных для хранения Тестов Тьюринга}  
База данных для хранения Тестов Тьюринга будет реализована с использованием СУБД SurrealDB. В базе данных будет существовать 7 сущностей и 7 типов отношений. ER-диаграма сущностей этой базы данных представлена на рисунке.  

\img{100mm}
{er}
{Диаграмма сущностей и отношений базы данных.}

Поля вершины Interrogator: 
\begin{enumerate}
    \item UniqueID --- уникальный идентификатор, присваемый сущности-субъекту; будет использоваться чтобы однозначно идентифицировать сущности, способные принимать решения в системе.  
    \item Name --- имя <<дознавателя>>.
    \item Gender --- пол <<дознавателя>>.
    \item Age --- возраст <<дознавателя>>.
    \item Nationality --- национальность <<дознавателя>>;
\end{enumerate}
Поля вершины Computer:
\begin{enumerate}
    \item UniqueID --- уникальный идентификатор, присваемый сущности-субъекту; будет использоваться чтобы однозначно идентифицировать сущности, способные принимать решения в системе.  
    \item Model --- Модель ИИ, проходившая тест
    \item DevelopedBy --- Разработчики указанной ИИ
\end{enumerate}
Поля вершины Human:
\begin{enumerate}
    \item UniqueID --- уникальный идентификатор, присваемый сущности-субъекту; будет использоваться чтобы однозначно идентифицировать сущности, способные принимать решения в системе;  
    \item Name --- имя человека;
    \item Gender --- пол человека.
    \item Age --- возраст человека.
    \item Nationality --- национальность человека;
\end{enumerate}
Поля вершины Answer:
\begin{enumerate}
    \item ItemID --- уникальный идентификатор, присваемый сущности-объекту; будет использоваться чтобы однозначно идентифицировать сущности, которые являются производными от объектов.
    \item AnswerText --- Текст Ответа.
\end{enumerate}
Поля вершины Question:
\begin{enumerate}
    \item ItemID --- уникальный идентификатор, присваемый сущности-объекту; будет использоваться чтобы однозначно идентифицировать сущности, которые являются производными от объектов.
    \item QuestionText --- Текст Вопроса.
\end{enumerate}
Поля вершины Verdict:
\begin{enumerate}
    \item ItemID --- уникальный идентификатор, присваемый сущности-объекту; будет использоваться чтобы однозначно идентифицировать сущности, которые являются производными от объектов.
    \item Correct --- Верен ли вердикт, выданный дознаватель.
\end{enumerate}
Поля Вершины Session:
\begin{enumerate}
    \item SessionID --- уникальный идентификатор, присваемый сессии.
    \item TimeFrame --- период времени, отведенный на ответ на вопрос
    \item TimeSpent --- продолжительность сессии.
\end{enumerate}
Поля ребер :AnsweredBy и :Follows:
\begin{enumerate}
    \item Order --- порядковый номер вопроса/ответа/вердикта в системе
\end{enumerate}

\subsection{Проектирование базы данных кэширования}

База данных кэширования будет реализована с помощью использования СУБД Tarantool. 
В базе данных будут полностью продублированны таблицы (в виде спейсов) из хранилища тестов Тюринга. 
Первичным ключом будет являться поле с уникальным идентификатором этих таблиц (id). 
При запросе данных у приложения, будет проводиться проверка, присутствует ли запись в кэше. 
Если запись присутствует, запрос к базе данных производиться не будет и будут возвращены данные из кэша. 
В противном случае, будет произведен запрос к базе данных.
Все спейсы будут созданы на основе движка memtx, хранящего все данные в оперативной памяти. 
Персистентность данных будет обеспечивается при помощи ведения журнала транзакция и системы «снимков» текущего
состояния кэша. 
Эти технологии помогут решить проблему «холодного»старта базы данных кэширования.
