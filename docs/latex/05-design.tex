\section{Конструкторский раздел}
В данном разделе представлены этапы проектирования выделенных в  
предыдущем разделе баз данных, нужных для решения задачи

\subsection{Проектирование базы данных для хранения Тестов Тьюринга}  
База данных для хранения Тестов Тьюринга будет реализована с использованием СУБД SurrealDB. В базе данных будет существовать 7 сущностей и 7 типов отношений. ER-диаграма сущностей этой базы данных представлена на рисунке.  

\img{100mm}
{er}
{Диаграмма сущностей и отношений базы данных.}

Поля вершины \texttt{Interrogator}: 
\begin{enumerate}
    \item \texttt{UniqueID} --- уникальный идентификатор, присваемый сущности-субъекту; будет использоваться чтобы однозначно идентифицировать сущности, способные принимать решения в системе;
    \item \texttt{Name} --- имя <<эксперта>>;
    \item \texttt{Gender} --- пол <<эксперта>>;
    \item \texttt{Age} --- возраст <<эксперта>>;
    \item \texttt{Nationality} --- национальность <<эксперта>>.
\end{enumerate}

Данная таблица отвечает за хранение данных, связанных с экспертом, проводящим эксперимент.
Эта таблица связана со следующими таблицами:
\begin{itemize}
    \item[$-$] \texttt{Session} --- через отношение \texttt{:ParticipateIn};
    \item[$-$] \texttt{Answer} --- через отношение \texttt{:Asks};
    \item[$-$] \texttt{Verdict} --- через отношение \texttt{:Gives}.
\end{itemize}

Поля вершины \texttt{Computer}:
\begin{enumerate}
    \item \texttt{UniqueID} --- уникальный идентификатор, присваемый сущности-субъекту; будет использоваться чтобы однозначно идентифицировать сущности, способные принимать решения в системе;
    \item \texttt{Model} --- Модель ИИ, проходившая тест;
    \item \texttt{DevelopedBy} --- Разработчики указанной ИИ.
\end{enumerate}

Данная таблица отвечает за хранение данных, связанных с компьютером, участвующем в эксперименте. 
Эта таблица связана со следующими таблицами:
\begin{itemize}
    \item[$-$] \texttt{Session} --- через отношение \texttt{:ParticipateIn};
    \item[$-$] \texttt{Answer} --- через отношение \texttt{:Gives};
    \item[$-$] \texttt{Verdict}. --- через отношение \texttt{:Mentions}.
\end{itemize}

Поля вершины \texttt{Human}:
\begin{enumerate}
    \item \texttt{UniqueID} --- уникальный идентификатор, присваемый сущности-субъекту; будет использоваться чтобы однозначно идентифицировать сущности, способные принимать решения в системе;  
    \item \texttt{Name} --- имя человека;
    \item \texttt{Gender} --- пол человека;
    \item \texttt{Age} --- возраст человека; 
    \item \texttt{Nationality} --- национальность человека.
\end{enumerate}

Данная таблица отвечает за хранение данных, связанных с человеком, участвующем в эксперименте. 
Эта таблица связана со следующими таблицами:
\begin{itemize}
    \item[$-$] \texttt{Session} --- через отношение \texttt{:ParticipateIn};
    \item[$-$] \texttt{Answer} --- через отношение \texttt{:Gives};
    \item[$-$] \texttt{Verdict} --- через отношение \texttt{:Mentions}.
\end{itemize}

Поля вершины \texttt{Answer}:
\begin{enumerate}
    \item \texttt{ItemID} --- уникальный идентификатор, присваемый сущности-объекту; будет использоваться чтобы однозначно идентифицировать сущности, которые являются производными от объектов.
    \item \texttt{AnswerText} --- текст ответа.
\end{enumerate}

Данная таблица отвечает за хранение данных, связанных с ответами, данными в эксперименте. 
Следует отметить, что ответы, данные на протяжении всех экспериментов, являются уникальными сущностями, или иным словами, в базе данных нет двух одинаковых ответов на любой из вопросов.
Данная особенность преследует цель показать связь между вопросами и ответами и как различные вопросы могут привести к одним и тем же ответам, либо же, как компьютер и человек в экперименте могут дать одинаковый ответ.

Эта таблица связана со следующими таблицами:
\begin{itemize}
    \item[$-$] \texttt{Session} --- через отношение \texttt{:Includes};
    \item[$-$] \texttt{Answer} --- через отношения \texttt{:AnsweredBy} и \texttt{:Follows};
    \item[$-$] \texttt{Computer} --- через отношение \texttt{:Gives};
    \item[$-$] \texttt{Human} --- через отношение \texttt{:Gives};
    \item[$-$] \texttt{Verdict} --- через отношение \texttt{:Follows}.
\end{itemize}

Поля вершины \texttt{Question}:
\begin{enumerate}
    \item \texttt{ItemID} --- уникальный идентификатор, присваемый сущности-объекту; будет использоваться чтобы однозначно идентифицировать сущности, которые являются производными от объектов.
    \item \texttt{QuestionText} --- текст вопроса.
\end{enumerate}

Данная таблица отвечает за хранение данных, связанных с вопросами, данными в эксперименте экспертом. 
Следует отметить, что вопросы, данные на протяжении всех экспериментов, как и ответы, упомянутые выше, являются уникальными сущностями, или иным словами, в базе данных нет двух одинаковых вопросов.
Данная особенность преследует цель показать связь между вопросами и ответами и как на один вопрос можно привести множество различных ответов.

Эта таблица связана со следующими таблицами:
\begin{itemize}
    \item[$-$] \texttt{Session} --- через отношение \texttt{:Includes};
    \item[$-$] \texttt{Answer} --- через отношения \texttt{:AnsweredBy} и \texttt{:Follows};
    \item[$-$] \texttt{Interrogator} --- через отношение \texttt{:Asks}.
\end{itemize}

Поля вершины \texttt{Verdict}:
\begin{enumerate}
    \item \texttt{ItemID} --- уникальный идентификатор, присваемый сущности-объекту; будет использоваться чтобы однозначно идентифицировать сущности, которые являются производными от объектов;
    \item \texttt{Correct} --- Верен ли вердикт, выданный экспертом.
\end{enumerate}

Данная таблица отвечает за хранение данных, связанных с вердиктами, данными экспертами по окончанию экспериментов.
После любого данного ответа, эксперт может закончить эксперимент и выдать свой вердикт, кто является компьютером, а кто человеком.

Эта таблица связана со следующими таблицами:
\begin{itemize}
    \item[$-$] \texttt{Session} --- через отношение \texttt{:Includes};
    \item[$-$] \texttt{Answer} --- через отношение \texttt{:Follows};
    \item[$-$] \texttt{Interrogator} --- через отношение \texttt{:Gives};
    \item[$-$] \texttt{Computer} --- через отношение \texttt{:Mentions};
    \item[$-$] \texttt{Human} --- через отношение \texttt{:Mentions}.
\end{itemize}

Поля вершины \texttt{Session}:
\begin{enumerate}
    \item \texttt{SessionID} --- уникальный идентификатор, присваемый сессии;
    \item \texttt{TimeFrame} --- период времени, отведенный на ответ на вопрос;
    \item \texttt{TimeSpent} --- продолжительность сессии.
\end{enumerate}

Данная таблица отвечает за хранение данных, связанных с различными экспериментами.
Данная таблица является свеого рода \texttt{meta}-таблицей, по связи с которой можно получить данные о всех сущностях, участвующими в эксперименте.

Эта таблица связана со следующими таблицами:
\begin{itemize}
    \item[$-$] \texttt{Question} --- через отношение \texttt{:Includes};
    \item[$-$] \texttt{Answer} --- через отношение \texttt{:Includes};
    \item[$-$] \texttt{Answer} --- через отношение \texttt{:Includes};
    \item[$-$] \texttt{Interrogator} --- через отношение \texttt{:ParticipateIn};
    \item[$-$] \texttt{Computer} --- через отношение \texttt{:ParticipateIn};
    \item[$-$] \texttt{Human} --- через отношение \texttt{:ParticipateIn}.
\end{itemize}

Поля рёбер \texttt{:AnsweredBy} и \texttt{:Follows}:
\begin{enumerate}
    \item \texttt{Order} --- порядковый номер вопроса/ответа/вердикта в системе
\end{enumerate}

Особенность \texttt{SurrealDB} заключается в том, что отношения также могут иметь дополнительные поля, характеризующие их.
Поле \texttt{Order} необходимо для построения контекста ответов/вопросов поскольку ответ может различаться от того, какие ответы были даны ранее.

\subsection*{Вывод}
В данном разделе были представлены этапы проектирования баз данных и рассмотрены особенности используемой СУБД на архитектурном уровне.
