\section{Конструкторский раздел}
В данном разделе представлены этапы проектирования выделенных в предыдущем разделе базы данных, нужной для решения задачи.

\subsection{Проектирование базы данных для хранения Тестов Тьюринга}  
База данных для хранения Тестов Тьюринга будет реализована с использованием СУБД SurrealDB.
В базе данных будет существовать 7 таблиц и 7 типов отношений. 
Схема разработанной базы данных представлена на рисунке~\ref{img:er_new}.

\img{125mm}
{er_new}
{Схема разработанной базы данных.}

Поля таблицы \texttt{Interrogator}: 
\begin{enumerate}
    \item \texttt{UniqueID} --- уникальный идентификатор --- \texttt{UUID};
    \item \texttt{Name} --- имя <<эксперта>> --- строка;
    \item \texttt{Gender} --- пол <<эксперта>> --- строка;
    \item \texttt{Age} --- возраст <<эксперта>> --- целое число;
    \item \texttt{Nationality} --- национальность <<эксперта>> --- строка.
\end{enumerate}

Данная таблица отвечает за хранение данных, связанных с экспертом, проводящим эксперимент.
Эта таблица связана со следующими таблицами:
\begin{itemize}
    \item[$-$] \texttt{Session} --- через отношение \texttt{:ParticipateIn};
    \item[$-$] \texttt{Answer} --- через отношение \texttt{:Asks};
    \item[$-$] \texttt{Verdict} --- через отношение \texttt{:Gives}.
\end{itemize}

Поля таблицы \texttt{Computer}:
\begin{enumerate}
    \item \texttt{UniqueID} --- уникальный идентификатор --- \texttt{UUID};
    \item \texttt{Model} --- Модель ИИ, проходившая тест --- строка;
    \item \texttt{DevelopedBy} --- Разработчики указанной модели ИИ --- строка.
\end{enumerate}

Данная таблица отвечает за хранение данных, связанных с компьютером, участвующем в эксперименте. 
Эта таблица связана со следующими таблицами:
\begin{itemize}
    \item[$-$] \texttt{Session} --- через отношение \texttt{:ParticipateIn};
    \item[$-$] \texttt{Answer} --- через отношение \texttt{:Gives};
    \item[$-$] \texttt{Verdict}. --- через отношение \texttt{:Mentions}.
\end{itemize}

Поля таблицы \texttt{Human}:
\begin{enumerate}
    \item \texttt{UniqueID} --- уникальный идентификатор --- \texttt{UUID};  
    \item \texttt{Name} --- имя человека --- строка;
    \item \texttt{Gender} --- пол человека --- строка;
    \item \texttt{Age} --- возраст человека --- строка; 
    \item \texttt{Nationality} --- национальность человека --- строка.
\end{enumerate}

Данная таблица отвечает за хранение данных, связанных с человеком, участвующем в эксперименте. 
Эта таблица связана со следующими таблицами:
\begin{itemize}
    \item[$-$] \texttt{Session} --- через отношение \texttt{:ParticipateIn};
    \item[$-$] \texttt{Answer} --- через отношение \texttt{:Gives};
    \item[$-$] \texttt{Verdict} --- через отношение \texttt{:Mentions}.
\end{itemize}

Поля таблицы \texttt{Answer}:
\begin{enumerate}
    \item \texttt{ItemID} --- уникальный идентификатор --- \texttt{UUID}.
    \item \texttt{AnswerText} --- текст ответа --- строка.
\end{enumerate}

Данная таблица отвечает за хранение данных, связанных с ответами, данными в эксперименте. 
Следует отметить, что ответы, данные на протяжении всех экспериментов, являются уникальными сущностями, или, иными словами, в базе данных нет двух одинаковых ответов на любой из вопросов.
Данная особенность преследует цель показать связь между вопросами и ответами, и как различные вопросы могут привести к одним и тем же ответам, либо же, как компьютер и человек в экперименте могут дать одинаковый ответ.

Эта таблица связана со следующими таблицами:
\begin{itemize}
    \item[$-$] \texttt{Session} --- через отношение \texttt{:Includes};
    \item[$-$] \texttt{Answer} --- через отношения \texttt{:AnsweredBy} и \texttt{:Follows};
    \item[$-$] \texttt{Computer} --- через отношение \texttt{:Gives};
    \item[$-$] \texttt{Human} --- через отношение \texttt{:Gives};
    \item[$-$] \texttt{Verdict} --- через отношение \texttt{:Follows}.
\end{itemize}

Поля таблицы \texttt{Question}:
\begin{enumerate}
    \item \texttt{ItemID} --- уникальный идентификатор --- \texttt{UUID}.
    \item \texttt{QuestionText} --- текст вопроса --- строка.
\end{enumerate}

Данная таблица отвечает за хранение данных, связанных с вопросами, данными в эксперименте экспертом. 
Следует отметить, что вопросы, данные на протяжении всех экспериментов, как и ответы, упомянутые выше, являются уникальными сущностями, или, иными словами, в базе данных нет двух одинаковых вопросов.
Данная особенность преследует цель показать связь между вопросами и ответами, и как на один вопрос можно привести множество различных ответов.

Эта таблица связана со следующими таблицами:
\begin{itemize}
    \item[$-$] \texttt{Session} --- через отношение \texttt{:Includes};
    \item[$-$] \texttt{Answer} --- через отношения \texttt{:AnsweredBy} и \texttt{:Follows};
    \item[$-$] \texttt{Interrogator} --- через отношение \texttt{:Asks}.
\end{itemize}

Поля таблицы \texttt{Verdict}:
\begin{enumerate}
    \item \texttt{ItemID} --- уникальный идентификатор --- \texttt{UUID};
    \item \texttt{Correct} --- Верен ли вердикт, выданный экспертом --- ложь / истина.
\end{enumerate}

Данная таблица отвечает за хранение данных, связанных с вердиктами, данными экспертами по окончанию экспериментов.
После любого данного ответа, эксперт может закончить эксперимент и выдать свой вердикт, кто является компьютером, а кто человеком.

Эта таблица связана со следующими таблицами:
\begin{itemize}
    \item[$-$] \texttt{Session} --- через отношение \texttt{:Includes};
    \item[$-$] \texttt{Answer} --- через отношение \texttt{:Follows};
    \item[$-$] \texttt{Interrogator} --- через отношение \texttt{:Gives};
    \item[$-$] \texttt{Computer} --- через отношение \texttt{:Mentions};
    \item[$-$] \texttt{Human} --- через отношение \texttt{:Mentions}.
\end{itemize}

Поля таблицы \texttt{Session}:
\begin{enumerate}
    \item \texttt{SessionID} --- уникальный идентификатор --- \texttt{UUID};
    \item \texttt{TimeFrame} --- период времени, отведенный на ответ на вопрос --- время;
    \item \texttt{TimeSpent} --- продолжительность сессии --- время.
\end{enumerate}

Данная таблица отвечает за хранение данных, связанных с различными экспериментами.
Данная таблица является свеого рода мета-таблицей, по связи с которой можно получить данные о всех сущностях, участвующих в эксперименте.

Эта таблица связана со следующими таблицами:
\begin{itemize}
    \item[$-$] \texttt{Question} --- через отношение \texttt{:Includes};
    \item[$-$] \texttt{Answer} --- через отношение \texttt{:Includes};
    \item[$-$] \texttt{Answer} --- через отношение \texttt{:Includes};
    \item[$-$] \texttt{Interrogator} --- через отношение \texttt{:ParticipateIn};
    \item[$-$] \texttt{Computer} --- через отношение \texttt{:ParticipateIn};
    \item[$-$] \texttt{Human} --- через отношение \texttt{:ParticipateIn}.
\end{itemize}

Поля рёбер \texttt{:AnsweredBy} и \texttt{:Follows}:
\begin{enumerate}
    \item \texttt{Order} --- порядковый номер вопроса/ответа/вердикта в системе - целое число
\end{enumerate}

% Пример SQL запроса, утилизирующий множественные JOIN-операции: 

Особенность \texttt{SurrealDB} заключается в том, что отношения также могут иметь дополнительные поля, характеризующие их.
Поле \texttt{Order} необходимо для построения контекста ответов/вопросов поскольку ответ может различаться от того, какие ответы были даны ранее.

\subsection{Структура разрабатываемого ПО}
Предполагается, что разрабатываемый проект является одним цельным \texttt{Electron}-подобным~\cite{electron} приложением.
Серверная часть и графический интерфейс упаковывается в единый бинарный файл, предоставляя возможность создать различные версии приложения под различные операционные системы.
В общем смысле, серверная часть коммуницирует с базой данных, доставляя результат к графическому интерфейсу в рамках единого приложения.

Общая схема архитектура приложения представлена на рисунке~\ref{img:arch}

\img{75mm}
{arch}
{Схема архитектуры приложения}


\subsection*{Вывод}
В данном разделе были представлены проектирование базы данных, рассмотрены особенности используемой СУБД на архитектурном уровне и была показана структура разрабатываемого ПО.
