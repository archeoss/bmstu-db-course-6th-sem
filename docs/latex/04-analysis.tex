\section{Аналитический раздел}

В данном разделе описана структура теста Тьюринга.  
Представлен анализ способов хранения данных и систем управления базами данных. 

\subsection{Анализ предметной области}
Тест Тьюринга --- это метод оценки способности машины производить интеллектуальные действия, сравнивая ее поведение с поведением человека в решении задач. 
На рисунке~\ref{img:turing} представлено схематическое изображение этого экперимента.

\img{100mm}
{turing}
{Тест Тьюринга}

Тьюринг в своей работе~\cite{10.1093/mind/LIX.236.433} описывает следующий вид игры.
Предположим, что у нас есть человек, машина и эксперт. 
Эксперт находится в комнате, отделенной от другого человека и машины. 
Цель игры состоит в том, чтобы эксперт определил, кто из двух является человеком, а кто машиной. 
Эксперт знает человека и машину по меткам <<X>> и <<Y>> --- но, по крайней мере в начале игры, не знает, кто из них человек и кто -- машина --- и в конце игры он должен сказать либо «X --- это человек, а Y --- машина», либо «X --- это машина, а Y --- человек».
Эксперту разрешается задавать человеку и машине вопросы следующего вида: <<Скажите, пожалуйста, X, играет ли X в шахматы?>> Кто бы из машины и другого человека ни был X, он должен отвечать на вопросы, адресованные X.
Цель машины состоит в том, чтобы попытаться заставить эксперта ошибочно заключить, что машина --- это другой человек; цель другого человека состоит в том, чтобы попытаться помочь эксперту правильно идентифицировать машину~\cite{sep-turing-test}.

Следует отметить, что во времена Тьюринга, было ограничение, что ответы поступали через ограниченные врменные рамки, поскольку время ответа компьютера было гораздо больше, чем у человека. 
Сегодня это ограничение сохраняется, однако из-за обратного: реакция компьютера быстрее, чем реакция человека~\cite{10.1093/mind/LIX.236.433}.

\subsection{Способы хранения данных}

Для решения задачи хранения теста Тьюринга необходимо хранить следующие данные:
\begin{enumerate}
  \item Данные о человеке, машине и эксперте;
  \begin{enumerate}
    \item Для человека:
      \begin{itemize}
        \item[$-$] Имя;
        \item[$-$] Пол;
        \item[$-$] Возраст;
        \item[$-$] Национальность.
      \end{itemize}
    \item Для машины:
      \begin{itemize}
        \item[$-$] Модель Искусственного Интеллекта;
        \item[$-$] Разработчик модели.
      \end{itemize}
    \item Для эксперта:
      \begin{itemize}
        \item[$-$] Имя;
        \item[$-$] Пол;
        \item[$-$] Возраст;
        \item[$-$] Национальность.
      \end{itemize}
  \end{enumerate}
  \item Данные о заданных вопросах и полученных ответах, а также вердикт эксперта c предположением о сущностях ответчиков;
    \begin{itemize}
      \item[$-$] Для вопросов --- текст вопроса.
      \item[$-$] Для ответов --- текст ответа.
      \item[$-$] Для вердикта --- корректность вердикта.
    \end{itemize}
  \item Данные об отдельном эксперименте:
    \begin{itemize}
      \item[$-$] Длительность проведения эксперимента;
      \item[$-$] Время, отведенное на дачу ответов субъектами.
    \end{itemize}
  \item Данные, о связях между сущностями:
    \begin{itemize}
      \item[$-$] Последовательность вопросов - ответов, оканчивающаяся вердиктом;
      \item[$-$] Связь между ответом и теми, кто его дал;
      \item[$-$] Связь между вопросом и теми, кто его задал;
      \item[$-$] Связь между вердиктом, тем кто его задал, и относительно каких субъектов он был вынесен;
      \item[$-$] Связь между экспериментом и всеми сущностями включенными в данный эксперимент.
    \end{itemize}
\end{enumerate}

На рис.~\ref{img:chen} изображена ER-диаграмма предметной области.

\img{200mm}
{chen}
{ER-диаграмма предметной области}

\clearpage

Поскольку в конце теста выносится вердикт о том, является ли отвечающий машиной или человеком, необходимо также хранить, какие ответы были даны, в каком порядке и каким актором. 
Эти данные должны быть доступны для обработки и сравнения в процессе игры.
 
Один из способов хранения данных --- это использование реляционных баз данных. В этом случае можно создать таблицы для каждого объекта (люди, машины и эксперты, вопросы и ответы, и т.д.) и связать их отношениями. 
Например, таблицы <<Person>>, <<Computer>> и <<Interrogator>> могут быть связаны через внешние ключи. 
Это является надежным и проверенным способом хранения данных.

Однако более эффективный способ хранения сильносвязанных данных --- это использование графовых баз данных (GDB), таких как \texttt{SurrealDB} или \texttt{Neo4j}~\cite{DBLP:journals/corr/LinMPS16}. 
В этом случае каждый объект может быть представлен узлом графа, а отношения между объектами --- ребрами графа. Схематическое изображение графовой модели данных можно увидеть на рисунке~\ref{img:graph_new}.

\img{100mm}
{graph_new}
{Представление базы данных в виде графа}

Ключевым понятием такой базы данных является граф (или ребро, или взаимосвязь).
Граф связывает элементы данных в хранилище с набором узлов и ребер, причем ребра представляют отношения между узлами. 
Отношения позволяют напрямую связывать данные в хранилище и во многих случаях извлекать их с помощью одной операции. 
Графовые базы данных удерживают отношения между данными в качестве приоритета.
Запросы по отношениям проходят быстро, потому что они постоянно хранятся в базе данных в качестве отдельных сущностей.
Отношения можно визуализировать с помощью графов, что делает их полезными для сильно взаимосвязанных данных~\cite{graph-exp}.
Поскольку графовая модель данных более естественным образом отображает связи между объектами, это делает ее более подходящей, чем реляционные базы данных, для задач, связанных с анализом связей и отношений между данными.
В графовых базах данных нет необходимости использовать сложные \texttt{JOIN}-запросы, что может существенно упростить запросы к данным по отношениям.

Графовые базы данных также обеспечивают быстрый доступ к данным по отношениям, что делает их эффективными при работе с глубоко связанными данными. Они также позволяют легко добавлять новые данные в граф без необходимости изменения схемы базы данных.
Однако, реляционные базы данных обладают более высокой надежностью и могут обеспечивать лучшую производительность при выполнении сложных запросов, особенно если используются правильно настроенные индексы.

Таким образом, выбор между реляционными базами данных и графовыми зависит от конкретных требований проекта.
В контексте данной работы, графовая модель подходит больше, чем реляционная, потому что тест Тьюринга включает в себя множество связей между объектами (человек, машина, эксперт, вопросы и ответы и т.д.), которые можно представить в виде графа.

Графовая модель также становится еще более привлекательной, если вспомнить какая идея была обозначена в начале данной работы: создание инструмента для улучшения искусственного интеллекта.
Результаты тестов Тьюринга, которые будут храниться в графовой базе данных, можно будет использовать для эффективного, по сравнению с традиционными реляционными базами данных, обучения ИИ. 
Эффективность достигается за счет того, что данные, используемые для обучения ИИ, представлены в виде графа, а не в виде стандартных <<табличных>> значений~\cite{besta2022neural}. 
Таким образом, схожая структура данных внутри БД поможет разработать более гибкую и быстродействующую систему при меньших затраченных ресурсах.

На рис.~\ref{img:graph-bench} можно увидеть результаты сравнения 3 различных баз данных: реляционной (\texttt{PostgreSQL}), графовой (\texttt{Neo4j}) и мультимодельной (\texttt{ArangoDB}). 
Исходя из графика видно, что графовая, а также мультимодельная база данных обеспечивает лучшую производительность при выполнении запросов по отношениям (\texttt{JOIN}-запросов), а также \texttt{PROJECTION}-запросов. 
Скорость графовой и мультимодельной баз данных относительно реляционной по \texttt{PROJECTION}-запросам объясняется особенностью \texttt{key-value} хранилищ, лежащих в основе графовой и мультимодельной баз данных.
При этом, реляционная база данных обеспечивает лучшую производительность при выполнении операций, не связанных с отношениями, а именно \texttt{ORDER BY} и \texttt{AGGREGATION}.

\clearpage

\img{100mm}
{graph-bench}
{Сравнение времени работы различных баз данных над атомарными операциями.}

\subsection{Ролевая модель}
В системе определены четыре возможные роли, ограничивающие доступ к получению/добавлению информации.
\begin{enumerate}
  \item Эксперт --- пользователь, обладающий возможностью задавать новые вопросы подопытным, а также выносить вердикт на оснвое полученных решений, имеет доступ к прочтению всех данных в рамках сессии;
  \item Компьютер --- пользователь, обладающий возможностью дачи ответов на вопросы эксперта, имеет доступ к чтению своих прошлых ответов и вопросов к ним в рамках сессии;
  \item Человек --- пользователь, обладающий равными с комьютером правами и воможностями. Ввиду различия характеристик между человеком и компьютером, в системе необходимо отдельно хранить данные об этих сущностях, а также выделить отдельные роли;
  \item Администратор --- пользователь, обладающий возможностью добавления/удаления/изменения всех пользователей, сущностей и полей базы данных, а также просмотра всех данных в рамках базы данных.
\end{enumerate}

На рисунке~\ref{img:roles} представлена диаграмма ролей.

\img{160mm}
{roles}
{Диаграмма ролей}

% \clearpage

\subsection{Системы управления базами данных}

% Для выбора системы управления базами данных необходимо учитывать требования к производительности и масштабируемости приложения.
%
% Реляционные базы данных имеют высокую надежность и поддерживают ACID-свойства транзакций. Они также обеспечивают хорошую производительность при выполнении сложных запросов. Однако, они требуют дополнительного управления индексами и ключевыми полями.
%
% Графовые базы данных обеспечивают высокую производительность при работе с глубоко связанными данными.
% Граф связывает элементы данных в хранилище с набором узлов и ребер, причем сами ребра представляют отношения между узлами.
% Отношения позволяют напрямую связывать данные в хранилище и во многих случаях извлекать их с помощью одной операции.
% Отношения между данными в подобных системах имеют приоритет над самими данными, поэтому запрос по отношениям является крайне быстрой операцией, поскольку они постоянно хранятся в базе данных. 
% Также отношения можно визуализировать с помощью графов, что делает их полезными для сильно взаимосвязанных данных. 

\subsubsection{SurrealDB}

\texttt{SurrealDB} --- мультимодельная \texttt{NewSQL} база данных, которая работает в режиме полной схемы (\texttt{SCHEMAFULL}) или без схемы (\texttt{SCHEMALESS}), с таблицами, ссылками на записи между документами (без \texttt{JOIN}) и функциями моделирования базы данных на основе графов~\cite{surrealdb}.

Благодаря использованию \texttt{SurrealDB} особых методов сегментирования и репликации, становится возможным повысить производительность за счет распределения нагрузки между несколькими компьютерами~\cite{surrealarch}.

Также особая архитектура базы данных позволяет работать как в оперативной памяти (\texttt{in-memory}), на дисковом пространстве (\texttt{on-disk}) или как распределенная база данных, используя \texttt{TiKV}~\cite{tikv}.

Поскольку \texttt{SurrealDB} --- мультимодельная база данных, становятся также возможными классические реляционные методики проектирования баз данных, что повышает гибкость итоговой системы.

\subsubsection{Neo4j}

\texttt{Neo4j} --- это графовая база данных, которая позволяет хранить, управлять и анализировать связанные данные. 
Она была разработана с учетом графовой модели данных, в которой данные представлены в виде узлов (вершин) и связей (ребер)~\cite{neo4j}. 

Одним из преимуществ \texttt{Neo4j} является то, что она позволяет эффективно моделировать и анализировать сложные связи между данными, такие как в социальных сетях, географических картах и сетях предприятий.
Это делает ее очень полезной для приложений, которые требуют быстрого доступа к сложным связным данным и быстрой обработки запросов.

Однако поскольку \texttt{Neo4j} --- исключительно графовая база данных, хранение и получение данных без каких-либо связей друг с другом может вызвать проблемы с производительностью, вне зависимости от размера запроса~\cite{neocons}.

\subsection{ArangoDB}

Подобно \texttt{SurrealDB}, \texttt{ArangoDB} также является мультимодельной базой даных, однако при этом является \texttt{NoSQL} базой данных в противовес \texttt{NewSQL}.

\texttt{ArangoDB} ---  был разработан специально для обслуживания различных типов баз данных \texttt{NoSQL}. 
Таким образом, варианты использования ArangoDB могут включать одновременную разработку баз данных, ориентированных на ключ, граф или документ~\cite{arangodb}.

Отличительной чертой всех \texttt{NoSQL} баз данных является отсутствие возможности объявления схемы данных внутри таблицы, и \texttt{ArangoDB} не исключение. 
Несмотря на то, что у \texttt{ArangoDB} есть внешний инструмент валидации данных через \texttt{JSON}-схемы, данная возможность проигрывает в гибкости настройки полноценным схемам традиционных реляционных и \texttt{NewSQL} баз данных.
Для непосредственного сравнения, документация \texttt{SurrealDB}~\cite{surrealdb-doc} насчитывает порядка 30 базовых типов, спецификация \texttt{JSON}-схемы~\cite{json-type} насчитывает всего лишь 7.
\subsection{Выбор СУБД для решения задачи}

Среди графовых баз данных можно выделить три наиболее подходящие системы: \texttt{SurrealDB}, \texttt{Neo4j} и \texttt{ArangoDB}.
Эти три СУБД обеспечивают быстрый доступ к данным по отношениям, что делает их эффективными при работе с глубоко связанными данными.

Однако \texttt{SurrealDB} имеет дополнительные преимущества перед \texttt{Neo4j} и \texttt{ArangoDB}. 
В противовес \texttt{Neo4j}, она является мультимодельной базой данных, что позволяет эффективное хранение и получение несвязанных данных, например выгрузка всех экспертов, которые учавстовали в различных экспериментах.
В таких случаях, \texttt{Neo4j} может испытывать определенные проблемы связанные в первую очередь с производительностью.

Если сравнивать \texttt{SurrealDB} c \texttt{ArangoDB}, то \texttt{SurrealDB} предоставляет более гибкий инструмент настройки схемы базы данных.

\subsection*{Вывод}

В данном разделе:
\begin{itemize}
  \item[$-$] рассмотрена сущность и структура теста Тьюринга;
  \item[$-$] определены ключевые роли в рамках эксперимента, связанного с тестом Тьюринга;
  \item[$-$] проанализированы способы хранения информации для системы и выбраны оптимальные способы для решения поставленной задачи;
  \item[$-$] были рассмотрены два различных типа баз данных: реляционные и графовые;
  \item[$-$] было выявлено, что для решения задачи теста Тьюринга наиболее подходящей является графовая база данных, а конкретно \texttt{SurrealDB}.
\end{itemize}

