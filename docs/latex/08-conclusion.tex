\specsection{ЗАКЛЮЧЕНИЕ}
% \addcontentsline{toc}{specsection}{ЗАКЛЮЧЕНИЕ}
В ходе выполнения проекта, цель данного курсовой работы была достигнута, то есть был разработан программный продукт, позволяющий хранить результаты тестов Тьюринга.
В ходе выполнения части было установлено, что запросы по связям могут выполняться быстрее в графовых базах данных, нежели запросы с использованием множественных JOIN-запросов в реляционных базах данных.
При этом следует помнить, что значительные выигрыши по времени могут наблюдаться только при условии большого размера базы данных и сильной связанности сущностей внутри неё.

Для достижение цели был выполнен ряд различных задач. 
Так, в первую очередь, был проведен анализ предметной области, определены основные сущности системы и связи между ними. 
Определена ролевая модель итогового приложения.
Затем было проведено сравнение различных баз данных, был найден оптимальный вариант для решения поставленной задачи --- \texttt{SurrealDB}.
Была спроектирована база данных для хранения Тестов Тьюринга, для которой были определны 7 различных таблиц и 7 типов отношений.
В качестве средства для реализации программного обеспечения был выбран язык \texttt{Rust} ввиду его быстродействия и удобства работы с ресурсами системы.
Был разработаны серверная часть и графический интерфейс конечного приложения. Разработаны и проведены эксперименты по замеру времени работы программного обеспечения.

