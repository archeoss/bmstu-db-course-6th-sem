\specsection{ЗАКЛЮЧЕНИЕ}
% \addcontentsline{toc}{specsection}{ЗАКЛЮЧЕНИЕ}
В ходе выполнения проекта, цель данного курсовой работы была достигнута, то есть был разработан программный продукт, позволяющий эффективно хранить результаты тестов Тьюринга.
В ходе выполнения экспериментально-исследовательской части было установлено, что запросы по связям будут выполняться быстрее в графовых базах данных, нежели запросы с использованием множественных JOIN-запросов в реляционных базах данных.
При этом следует помнить, что значительные выигрыши по времени могут наблюдаться только при условии большого размера базы данных и сильной связанности сущностей внутри неё.

Для достижение цели были выполнены следующие задачи:
\begin{itemize}
    \item[$-$] проведен анализ предметной области --- Тест Тьюринга;
    \item[$-$] спроектирована архитектура программного обеспечения, на которой были отображены сущности и возможные связи между ними;
    \item[$-$] выбраны средства реализации программного обеспечения c подробным описанием их особенностей;
    \item[$-$] разработано программное обеспечение;
    \item[$-$] разработаны и проведены эксперименты по замеру времени работы программного обеспечения.
\end{itemize}

