\specsection{ВВЕДЕНИЕ}
% \addcontentsline{toc}{specsection}{Введение}

Фраза <<Тест Тьюринга>> наиболее правильно используется для обозначения предложения, сделанного Тьюрингом (1950) как способ решения вопроса о том, могут ли машины мыслить~\cite{sep-turing-test}. 
Согласно Тьюрингу, вопрос о том, могут ли машины мыслить, сам по себе <<слишком бессмыслен>>, чтобы заслуживать обсуждения~\cite{10.1093/mind/LIX.236.433}.
Для проведения теста Тьюринга используется программное обеспечение, которое имитирует человеческое поведение и должно убедить эксперта в том, что он общается с живым человеком~\cite{10.1093/mind/LIX.236.433}.
Результаты проведения тестов могут быть использованы для разработки и улучшения алгоритмов искусственного интеллекта~\cite{DBLP:journals/corr/abs-2106-11394}.

Примерами таких алгоритмов ИИ являются алгоритмы обработки естественных языков (Natural Language Processing --- NLP).
NLP --- это совокупность методов и техник, которые позволяют компьютерам анализировать, понимать и генерировать естественный язык~\cite{nlp}.
NLP используется в ряде приложений, включая автоматический перевод, распознавание речи и анализ текста~\cite{nlp}. 
Анализ результатов тестирования поможет в будущем улучшить данные модели, позволяя избегать различные грамматические, орфографические и смысловые ошибки. 

Модели от команды OpenAI и многие другие играют важную роль в развитии искусственного интеллекта~\cite{joublin2023glimpse}.
GPT-3 используется для создания художественной литературы, поэзии, пресс-релизов, кода, музыки, шуток, технических руководств и новостных статей~\cite{zong2022survey}, как предполагает Чалмерс~\cite{gpt3}, GPT-3 <<предлагает потенциальный бездумный путь к общему искусственному интеллекту>>. 
Но, конечно, GPT-3 даже не близок к прохождению теста Тьюринга: на глобальном уровне — учитывая значения нескольких предложений, абзацев или двустороннего диалога --- становится очевидным, что <<GPT-3 пишет текст, в продолжение введенных слов, без какого-либо понимания>>~\cite{gpt3-undersatnding}. 
У него нет здравого смысла~\cite{kojima2023large} или способности отслеживать объекты во время обсуждения~\cite{gpt3-track}. 
% Разработанная база данных для хранения и обработки результатов тестов Тьюринга --- это хороший инструмент для создания более развитых систем искусственного интеллекта. 

Цель данной работы --- разработать базу данных для хранения результатов проведения тестов Тьюринга. База данных должна содержать информацию о тестируемых программах, экспертах, результатах проведения тестов и другие данные, необходимые для анализа результатов.

Чтобы достигнуть поставленной цели, требуется решить следующие задачи: 
\begin{itemize}
    \item[$-$] Проанализировать предметную область.
    \item[$-$] Определить структуру базы данных и ее таблиц.
    \item[$-$] Выбрать модели данных и СУБД для реализации разрабатываемой базы данных.
    \item[$-$] Написать запросы для вставки, обновления и удаления данных в таблицах.
    \item[$-$] Провести исследование эффективности разработанной базы данных.
    \item[$-$] Реализовать функционал для получения результатов проведенных тестов, с возможностью фильтрации и сортировки по различным полям.
    \item[$-$] Разработать интерфейс пользователя для использования базы данных. 
\end{itemize}
