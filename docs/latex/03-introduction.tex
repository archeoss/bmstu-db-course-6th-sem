\specsection{ВВЕДЕНИЕ}
% \addcontentsline{toc}{specsection}{Введение}

Тест Тьюринга --- это способ определения способности машины производить интеллектуальные действия, неотличимые от действий человека. В современном мире тесты Тьюринга стали одним из ключевых инструментов для определения искусственного интеллекта. Эти тесты позволяют определить, насколько хорошо компьютер может имитировать разговор человека.
Для проведения теста Тьюринга используется программное обеспечение, которое имитирует человеческое поведение и должно убедить эксперта в том, что он общается с живым человеком. Результаты проведения тестов могут быть использованы для разработки и улучшения алгоритмов искусственного интеллекта. Это позволяет разработчикам алгоритмов искусственного интеллекта улучшать свои продукты и технологии.

Примерами таких алгоритмов являются алгоритмы обработки естественных языков (Natural Language Processing --- NLP). В общих словах --- это совокупность методов и техник, которые позволяют компьютерам анализировать, понимать и генерировать естественный язык. NLP используется в ряде приложений, включая автоматический перевод, распознавание речи и анализ текста. Анализ результатов тестирования поможет в будущем улучшить данные модели, позволяя избегать различные грамматические, орфографические и смысловые ошибки. 

Модели от команды OpenAI и многие другие играют важную роль в развитии искусственного интеллекта. GPT-3 неплохо справляется с созданием художественной литературы, поэзии, пресс-релизов, кода, музыки, шуток, технических руководств и новостных статей. Возможно, как предполагает Чалмерс (2020, Other Internet Resources), GPT-3 «предлагает потенциальный бездумный путь к общему искусственному интеллекту». Но, конечно, GPT-3 даже не близок к прохождению теста Тьюринга: на глобальном уровне — учитывая значения нескольких предложений, абзацев или двустороннего диалога — становится очевидным, что GPT-3 не понимает, о чем говорит. У него нет здравого смысла (Common Sense) или способности отслеживать объекты во время обсуждения. Разработанная база данных для хранения и обработки результатов тестов Тьюринга --- это хороший инструмент для создания более развитых систем искусственного интеллекта. 

Цель данной работы --- разработать базу данных для хранения результатов проведения тестов Тьюринга. База данных должна содержать информацию о тестируемых программах, экспертах, результатах проведения тестов и другие данные, необходимые для анализа результатов.

Чтобы достигнуть поставленной цели, требуется решить следующие задачи: 
\begin{itemize}
    \item[$-$] Определить структуру базы данных и ее таблиц.
    \item[$-$] Разработать модели данных для каждой таблицы.
    \item[$-$] Написать запросы для вставки, обновления и удаления данных в таблицах.
    \item[$-$] Реализовать функционал для получения результатов проведенных тестов, с возможностью фильтрации и сортировки по различным полям.
    \item[$-$] Разработать интерфейс пользователя для удобного использования базы данных. 
\end{itemize}
